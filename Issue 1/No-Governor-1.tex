\documentclass[12pt, onecolumn, letterpaper, oneside]{book}

\usepackage{authblk}
\usepackage[autostyle]{csquotes}
\usepackage{amsmath}
\usepackage{graphicx}
\usepackage{hyperref}
\usepackage[sc]{titlesec}

\usepackage[square,numbers,sectionbib]{natbib}
\usepackage{chapterbib}
\usepackage{titlesec}
\usepackage{indentfirst}

\titleformat{\chapter}[display]
  {\rmfamily\scshape\bfseries}{}{0pt}{\Large}
\titleformat{\section}[display]
  {\scshape}{}{0pt}{\large}
  
 \usepackage{scrextend}


\usepackage{suffix}

\newcommand\chapterauthor[1]{\authortoc{#1}\printchapterauthor{#1}}
\WithSuffix\newcommand\chapterauthor*[1]{\printchapterauthor{#1}}

\usepackage{fancyhdr}
 
\pagestyle{fancy}
\renewcommand{\chaptermark}[1]{\markboth{}{\uppercase{\itshape #1}}}
\fancyhf{}
\fancyhead[L]{\rightmark}
\fancyhead[R]{\thepage}
\renewcommand{\headrulewidth}{0pt}
 

\makeatletter
\newcommand{\printchapterauthor}[1]{%
  {\parindent0pt\vspace*{-25pt}%
  \linespread{1.1}\large\scshape#1%
  \par\nobreak\vspace*{35pt}}
  \@afterheading%
}
\newcommand{\authortoc}[1]{%
  \addtocontents{toc}{\vskip-10pt}%
  \addtocontents{toc}{%
    \protect\contentsline{chapter}%
    {\hskip1.3em\mdseries\scshape\protect\scriptsize#1}{}{}}
  \addtocontents{toc}{\vskip5pt}%
}
\makeatother

\title{No Governor}
\author{Vol. I, No. 1.\\Editor Robert Shea}
\date{Spring, 1975}

\fancyfoot[L]{Summer, 1974}
\fancyfoot[R]{No Governor}


%\setlength\parindent{0pt}
    \setlength{\columnsep}{1cm}

\begin{document}
\sloppy

\maketitle

\vspace*{\fill}

\begin{flushright}
\emph{Re-edited using \LaTeX $\,$ for the 'Patafizikhood of Eris Esoteric.\\
If you have any comments or detected any mistake, please contact us at:\\
anarchao--at--discordia--dot--fr.\\}
\end{flushright}

\begin{flushright}
\emph{Thanks to Mike Shea for giving us permission to do this.}
\end{flushright}

\vspace*{\fill}

\tableofcontents

\chapter*{Notes}
\chapterauthor*{Robert Shea}

\blockquote{``There is no governor present anywhere."
\par\begin{flushright} --- Chuang Tzu \end{flushright}
}

After looking and asking around, it appeared to me that no magazine strictly devoted to anarchist ideas and opinion now exists in this part of the world, and that there is a need for one. There are publications devoted to news, to discussions of practical experiences and to polemic, and all of them include some theoretical discussion. But no periodical is devoted just to discussing and developing the underlying ideas that unite --- or divide --- the anarchist movement. We need a fuller understanding of what the term anarchism means. Many who would agree with the sentiment, ``Off the government," have no idea that their own pet proposals might bring back government under a different name. Many are unaware of the problems that will probably arise in anarchist practice, the multiplicity of legitimately anarchistic points of view, the range of possible means to abolish government, the relative merits of the various options.\\
There's a need to express the old ideas that are still valid in new language, to reach contemporary society. There are unresolved questions which need to be further debated. There are new conditions, new opportunities and new dangers to which anarchist ideas must be adapted. To provide an arena in which the ideas and writings of anarchists can meet, be exchanged and clash, I undertake to publish \emph{No Governor}.\\
Many anarchists feel a great impatience with theory, with ideas, with words and discussions. And this is healthy. Words, ideologies, abstractions, generalizations have frequently been used by authorities as instruments of control over subject populations. In a sense, anarchism is an expression of the wordless, vital impulse of rebellion that leaps up irrepressibly in the human heart. Like Zen and Taoism, which are ultimately anarchistic in their implications, anarchism goes beyond words. Yet, anarchism needs words to define itself, to distinguish itself beyond any possibility of confusion from government-oriented political philosophies, to free itself from authoritarian tendencies. It's quite possible to be an unconscious anarchist, but the conscious anarchist has the considerable advantage of knowing what he or she is doing.\\
\emph{No Governor} needs four kinds of help:
\begin{enumerate}
\item It needs to be read. One of the most important responsibilities of an anarchist is self-development as an anarchist. The words on these pages will have life and value only if you let them into your mind. If their only function there is to arouse opposing thought, they will still have served their purpose.
\item It needs comment and criticism. After reading this issue, please write us about the magazine and about specific articles. Letters from readers will, I hope, be a very important part of all future issues.
\item It needs articles. What kind of articles? Well, as we've been saying, the focus in \emph{No Governor} will be on opinion and ideas. Intellectual, theoretical articles. The six pieces in this issue give a pretty good idea of what we will be looking for. And we'd like to publish articles commenting on current events from an anarchist point of view; articles describing practical experience in organization and agitation and drawing strategic conclusions; articles proposing anarchist solutions for some of society's current problems such as war, ecological destruction, crime, poverty, sexism, racism; articles discussing human psychology and sociology from an anarchist standpoint; articles suggesting anarchist strategies for changing society.
\item It needs money to help with the cost of printing and mailing. Anyone who can't pay for a copy of the magazine and who wants one, can get one issue just for the asking. Free sample copies will also be sent to anyone you designate. But there will be no such things as a free subscription. Each issue has to be requested individually, in order to get it free. From those who can pay, I ask 25 cents an issue or \$1 for four issues. Those who contribute articles to \emph{No Governor} will receive subscriptions of indefinite and unforeseeable length.
\end{enumerate}
It has been difficult to organize my time to get this first issue out, so it's labeled ``Summer, 1974" up to page [23] and ``Spring, 1975" thereafter. So this is actually the Spring, 1975 issue. I see reason to think that I am a better organized person now than I was last year, so I hope that the next issue will be Summer, 1975, and that I can follow a quarterly publishing schedule thereafter.\\
So, here is a magazine dedicated to the promotion of anarchist consciousness. My own biases as an anarchist are pacifistic and individualistic, but I will try to publish what I consider good anarchist writing, regardless of whether or not I agree with the position it takes. I take full responsibility for editorial judgments, including all the mistakes.

\chapter{Should We Cooperate?}
\chapterauthor{Jim Bumpas}

The above question is raised in the context of working with organizations formed upon non-, or maybe anti-anarchist perspectives in joint action upon projects of general popular interest and concern.\\
The gut-level reaction of many anarchists, especially in regard to liberal or Communist groups, is to trash them whenever they ask for our cooperation or when they try to recruit new members. We don't notice them most of the time, but when we do notice them, we first want to deal them the same kind of blow we want to deliver to the corporate establishment. This is all the more tempting because of their relative vulnerability compared to ITT of General Motors.\\
Under certain conditions, my advice is to cooperate with both Marxist-Leninist and liberal groups. My conditions may seldom be met, however. First of all, any cooperation must be only for the limited purpose of furthering a current popular project, such as agitation and organizing public protest (i.e., anti-war or anti-high prices demonstrations).\\
Secondly, the other organizations must recognize and tolerate the organizational integrity of the anarchists and not insist that participating anarchists render their perspectives invisible. Where Marxists or liberals are strong they usually try to insist that all who wish to work on a project which is ``their property" must join their organization and accept its dominant leadership.\\
This proprietary attitude towards popular issues leads them to some of the following tactics in order to compel subservience to their leadership: ``What are you, a wrecker? Do you want to cause disunity and lessen our chances of possible success?" Or, if there is already a coalition of Marxist-Leninist and liberal groups, they will still accuse you of ``wrecking," since anarchists will surely alienate some other members of the coalition. Don't give in to this tactic. The argument cuts both ways. Maintain your independence.\\
Of course, without anarchist organization, you are more or less at the mercy of this tactic. Your energy and commitment to the issue will lead you to contribute much of the effort which will strengthen the organization of the Marxist-Leninist group and increase the future dominance over the local movement.\\
A third condition must be tolerance by the other groups of your expression of anarchist perspectives and of your independent criticism of their perspectives and practice. This will again be described as wrecking, or some similar epithet, but it is necessary to maintain our integrity and to demonstrate that our cooperation is not slavish adherence to dogma which we abhor, but only an ad hoc cooperation for the purpose of the immediate project. Any other course of action would be dishonest and misleading to the very people who have generally misunderstood anarchism. If it is ``wrecking" it is only because Marxist-Leninist leaders fear the effect our perspectives might have on their otherwise docile sheep.\\
If these conditions are refused, it does not preclude independent action on the project, especially if it is important enough to you. If your anarchist group is well enough organized, you will attract independent support of your own, forcing the recognition of hostile groups who are working on similar projects. A group I was a member of in college was forced to take this course because local Marxist-Leninist groups were convinced they had a proprietary interest in the anti-war movement and in any strike-support activity. When our group began conducting our own activities against the war and  in support of various strikes, our strength in both numbers and energy expanded rapidly. Then the Marxist-Leninist and liberal groups condescendingly began to invite us to participate equally in ``their" demonstrations. Anarchist perspectives soon began to excite people who previously had equated anarchy with chaos, anarchism with terrorism, anarchists with bomb-throwers. Only the transient nature of student populations reduced the effectiveness of this group, which still exists in a semi-dormant state after six years.

\section*{Why cooperate?}

All right, now that I've answered the first question affirmatively, and have discussed some of the mechanics of cooperation, why go to all the trouble? Why not trash all enemies or probable enemies whenever and wherever we are confronted with them? The question contains a suggestion of its own answer. Our enemies, both potential and actual, are many and we are few. I alluded to another part of the answer at the beginning of this discussion. The corporate establishment which dominates our culture and society, and which controls the government and its means of repression, including the military and police forces, media, education, etc., are vastly more powerful at this point that all of the Marxist-Leninist and liberal groups who might wish to eradicate anarchists at some time in the future. For purely strategic purposes, cooperation with the enemies of our most powerful enemies is called for.\\
While such cooperation will inevitably strengthen the Marxist-Leninists and liberals relative to the corporate establishment, the conditions I've described should ensure that our own strength should grow apace, if not more rapidly than theirs. At least we will not be swallowed up whole as the Bolsheviks swallowed anarchists in Russia. I realize my conditions for cooperation (not collaboration) are similar to those conditions which operated in revolutionary Spain in 1936--39, but I suggest that these conditions were not a factor in the anarchist defeat there.\\
Cooperation with other opponents of the corporate establishment will help us to create the conditions for a greater flowering of our energies and perspectives. If we continue to allow the biggest oppressor to pit all of its opposition against one another, the status quo stays in balance. It is to our advantage to tip the status quo out of balance. That is my hypothesis. Any comments?

\chapter{The Greedy Gurus and the False ``New" Left}
\chapterauthor{Tom McNamara}

The American public has a long history of being preyed upon by salvation sellers. If you look back, you'll come across Bible-bangers of the stripe of Cotton Mather, Billy Sunday, Aimee Semple McPherson, Father Divine, Daddy Grace and many other less-known fleecers of the flock. The air-waves still reverberate with the booming voices of the fundamentalists, Oral Roberts (recently become ``respectable" since he has allied with conservative Methodism), Garner Ted Armstrong (who runs a multimillion-dollar church empire that rakes in countless millions a year), conk-haired Reverend Ike (who tells his black followers to get off welfare and ``God" will give them a Cadillac) and Nixon's official evangelist, Billy Graham (whose most recent worry was that his ``God" was going to destroy him with a cataclysm, which gives you an idea about how important he thinks he is).\\
Recently, particularly among the young offspring of America's muddled middle class, it has become stylish to rebel against all this. For about a decade, new left mysticism, fueled by a desire to end the Vietnamese war, was very much in vogue. It was a glamorous way of rebelling and there was a secular religiosity, very much unrecognized, in demonstrating, attending meetings, making flaming speeches about burning issues, writing for movement media, concocting ideologies to fight about with competitive groups. The mass media devoted oceans of space to it all as a sign of its liberalism and everyone involved convinced themselves their commitment to the revolution was heartfelt and lifelong. The disillusionment set in heavy as the 70s dawned and the goals of the pseudo-left revolutionaries suddenly turned into just another flavor of pie in the sky. To counteract the despair and nothingist feelings, to fill the void left by the seeming lack of anything real to commit themselves to, thousands of young people have turned to religious cultism of various unorthodox brands in order to absorb more mysticism and to get the answer to it all.\\
If ``religion is the opiate of the people," as Marx stated, these new synthetic cults are the cheapest, most adulterated kind of street-junk heroin. Their leaders are the most hypocritical charlatans and exploiters and, without fail, fit very snugly into the left cliche ``lackies of the imperialistic ruling class." They are exponents of acquiescence to the oppressive system, worshippers of the status quo, mystagogues of the most crass order, hocus-pocus artists who make palm-readers look like saints by comparison, and are the direct descendants of the carnival ripoff snake oil sellers and other mountebanks. In your modern time they've gussied up their acts with the trappings of scientism, in some instances, using the credibility of gullible millions to weave an electronic mythology of ESP, flying saucers and supposed contacts with spacemen to bilk the ignorant of millions upon millions of hard-earned dollars, creating a very luxurious lifestyle for themselves. Some, like two-ton butterball boy avatar Guru Maharaj Gee-whiz, even have the affrontery to state that since they are ``God" themselves, they deserve to ride in Rolls Royce automobiles and live like kings. They are not only the darlings of the mass media, but of a good part of the old new left as well.\\
By now we've all heard more than enough about the boy ``God" and his Divine Light Mission, which runs up a truly fantastic electric bill to produce the sleazy neon glaze that passes for illumination to the dark minds of his easily-impressed followers. As with the Moslems --- ``There is one God and Mohammed is his prophet" --- there is only one (thank whatever!) Gooroo Gee-whiz and Rennie Davis is his very vocal barker. Davis, star of the Chicago Seven trial and other movement happenings, enhoys an extension of his time in the limelight and his role of apologizer for the Gooroo and his various enterprises. Some people have an insatiable need for power trips and publicity, and the more absurd the proposition, the more challenging to their ability to rationalize their involvement and explain it. Anything as long as they are at or near the center of vast attention.\\
Even when the Gooroo was discovered trying to smuggle watches and illegal bundles of cash into India, Davis and the other mahatmas of this crass and materialistic faith racket were able to laugh it all off in one way or another. True believers will take anything. At least for a while, and most of them are guilt-ridden enough to feel much lighter after they are separated from their valuables by these smiling spiritual thieves and con-men. If this cunning theft were all these string-them-along artists were up to, it would be easy enough to write them off as somewhat intelligent economic criminals. But they are concerned even more with offering soporifics to the hungering minds of their sheeplike followers, blunting their human concerns with changing not only their own spiritual status in a real way bit in taking action to change the horrifying inequities in the real and tangible world around them. The closest the Divine Light Mission comes to social action is re-selling the goods ripped off from new believers in their Salvation-Army-like thrift store. (A thrift store is one in which destitute people are paid little or nothing to sell merchandise usually repaired by other unfortunates who are paid starvation wages, if anything, so that the exploiting organization can reap the profits with little actual overhead.)\\
Although Rennie Davis' promotion is currently the hottest show in town, there is no shortage of hip peddlers of instant enlightenment, meditation --- transcendental and otherwise --- brainwave adjustments, positivity courses, ESP lessons, psychedelic mysticisms, Scientology, UFO cultism, dope theologies, radical psychotherapies, cures and healings and various kinds of blessings, adjustments, both mental and physical, and a thousand and one variations on old and tried (and found worthless) schemes. What few stop to realize is that if the society they live in were not so atrociously rotten, all these vapid panaceas would not only be unnecessary but would be positively laughable. Only the true-believing political groupies, whose spectrum runs from the old and new lefts of various shapes and sizes to the Young Americans for Freedom and beyond even talk about social reform, and that's almost always after the revolution (their equivalent of heaven on earth). These groups have more in common with the pseudo-religions than their members could ever let themselves realize without dropping out. They require true belied, acceptance of almost everything on faith, adulation of all-knowing leaders (Marx, Stalin, William Buckley) and a total acceptance of authoritarianism. This is where anarchism differs radically from all of them. There are no authorities in anarchism and, typically, there is a healthy disrespect for any widely-held opinion.\\
Where the left has fallen down is exactly where the cultists have filled in, gaining converts by the droves. Although both groups put off satisfaction in the now until much later (``after the revolution" or ``in heaven"), the left political groups rigidly changed when the present power structure is overthrown along the explicit correct lines of their particular brand of absolute authoritarian socialism. Some are so blunt about their fanaticism as to sound like the hellish documents of the Nazis. \emph{The Second Page}, a newspaper published by a group in San Francisco, joyously states:
\blockquote{While everybody working towards successful revolution has a responsibility to deal with their personal difficulties as best they can, it is against the interest of the revolution for the revolutionary part to devote much energy to such matters. We don't have time to focus on inner group introspection and ``group-dynamics." We can't afford to concentrate on personal despair. We can't personally resolve the misery of a class fall, or the inequities of our society.}
Hold all the hurt in, folks, don't get blood on the party banner, the revolution will be all over in a hundred years, more or less. No wonder the lemmings desert such slave ships and run for the nearest spiritual snug harbor they can find!

\chapter{The Anarchist Movement --- Dead or Alive?}
\chapterauthor{Arlene Meyers}

Once we get beyond the level of personal hostility in the \emph{SRAF Bulletin}, I think it's time we begin a serious critique of the anarchist movement. The first question is: Are we really a revolutionary movement or not?\\
When non-anarchists express interest in the ideas of anarchist thinkers, the question they ask is, ``How does anarchism work?" When all we have to offer are books on historic struggles or our own splintered and embattled ranks as examples of anarchism at work, we don't make any positive statement on anarchism.\\
From my association with Chicago anarchists, I would say our primary problem is integrating our politics with our daily lives. Not a matter of assuming an identity as an ``anarchist" --- role-playing at best, confusing revolution with religion at worst --- but as a conscious and deliberate process of creating social change in our own heads. Fucked-up people create fucked-up revolutions. The shape and direction of our revolution is determined by the kind of people we are, how we are growing and changing as individuals, and how we relate to one another.\\
Because so many anarchists I know are basically alienated, anti-social types incapable of genuine interaction with others, the movement reflects this isolation. Many anarchists are well-read, even brilliant in their grasp of history and theory, but because they lack event minimal social skills, they are unable to translate their ideas into collective action. Collective activity (where possible) is centered within small, closed friendship groups. Because these groups are so closed off and ingrown, their level of energy and activity is very limited. Even when the friendship group seeks to involve others in its activity, it is unable to open up its internal social structure to include new members because to do so would require an openness and energy the group does not possess.\\
Thus, we have a closed cycle of alienation (from society at large), exclusive association (with like-minded individuals), and isolation (based upon these ingrown structures). At this point anarchism is not a social movement but an anti-social movement.\\
Chicago anarchists --- other than a few exclusive friendship groups --- do not even care to associate with one another, because our attempts at open meetings several years ago became lifeless weekly rituals which anyone could come into and freely disrupt everyone else (and frequently did). Individuals with something interesting to say (and well worth listening to), and people who wanted to learn about anarchism and people who simply desired a social group were all lumped together in one frustrated mass. The logical thing might have been to separate into specific groups, but since we really had no clear idea of our own needs, I think we feared separation might lead to further isolation. In the end the group disintegrated anyway --- lack of direction more than anything else, I think.\\
Even now, those of us involved in activity in Chicago do not have any sort of \emph{open}, regular forum where we can exchange ideas, offer mutual aid and support or even keep each other informed, and this lack of communication seriously threatens our present level of activity and our continued growth; Because of previous hassles with organizing ourselves, I understand the reluctance of many anarchists here to have open meetings again, but ignoring problems does not make the problems go away.\\
Good communication is the lifeline of non-hierarchical organization. In Chicago, lines of communication are the ``property" of specific publications, groups and privileged individuals (e.g., \emph{Industrial Worker}, \emph{Black Cross Bulletin}, \emph{Siren}, I.W.W., Solidarity Bookshop). It was only when I began publishing \emph{Siren} that I was allowed access to privileged information withing the movement itself through exchange papers, personal correspondence and other resources. Unfortunately, there was no way for me to pass this privileged information along to other people because my efforts in organizing an anarcho-feminist group here were met with both hostility and indifference, and infrequent Solidarity Bookshop meetings (when called) were confined to bookshop business itself, discussions on stock, bills, etc.\\
Aside from necessary security precautions, news and information should always be offered freely, and not withheld as a condition for inclusion in the group\footnote{I mean, trading labor and loyalty for access to vital information (as a lot of ``revolutionary" groups do) is not very revolutionary. Solidarity people didn't even offer this much --- they are \emph{so} exclusive, I have no idea what is their criterion for inclusion in the inner circle. Which means that if there is no interaction within the collective structure (and collective identity) there is usually a big turnover in personnel.} (e.g., payment of dues, membership, building organizational strength, etc.). Lately I've stayed away from groups involved in social action because they demand some sort of commitment in the form of membership, dues, voluntary labor, etc. before offering any information, and I refuse to involve myself with people who play these games. Don't expect people to become involved if you refuse to open up to \emph{them}. People do not become involved in revolutionary activity in order to be as isolated as they are elsewhere: we become involved because we seek new ideas and associations, not merely to exchange one form of alienated activity for another.\\
Too often, anarchist activity is confined to the individual's own particular level of energy and resources and this usually means a lot of fragmented, duplicated efforts. Collective activity requires a level of social consciousness and cooperation the self-centered egotist does not possess. Collective activity should not require a submersion of the ego, but it does require a free exchange of ideas and energy. Insecure people who are jealously concerned only with their own ego needs can never participate freely in collective activity, because they become competitive in their drive for ``property" --- in this case, power and recognition --- in order to assure themselves a secure place in the collective structure. It's one thing to work with dynamic and energetic individuals; it's another thing to work with insecure people who engage in ruthless competition; backbiting and power-plays in their endless search for status and prestige.\\
Aside from a few enlightened individuals, too many anarchists I know are closed to struggle because they prefer to project a cool image based upon knowledge and experience they do not possess. This is revolutionary role-playing, and it's a mind-fuck. When people claim to be anarchists, and only offer theoretical abstractions or vague generalizations in response to specific problems, I know I'm dealing with people who not only have no answers but who refuse to commit themselves to an honest search for answers because it might reveal their inadequacy as ``revolutionaries." It is this unwillingness to struggle, and this moral cowardice in dealing with real problems which seriously flaws the anarchist movement of today.\\
Too many of us see ``struggle" solely in militaristic terms: bombs, bullets, sabotage, terrorism, etc. Yet struggle also means dealing with one's own fears and insecurities, such as racist or sexist attitudes. For many of us, it is easier to confront a line of riot cops than to deal with the complexities of personal relationships, yet power games and authoritarian attitudes begin in our own little heads, and our heads are really the first battleground where we initiate the struggle against the state. We must free ourselves of our fears and hypocrisies and power relationships because only a free people can create a revolutionary movement capable of transforming patriarchal, authoritarian structures into the free society we desire.\\
A revolutionary movement is only a vehicle to freedom, and not freedom itself. It must be both open and viable; a place to discuss new ideas and to experiment with new forms and structures; a vehicle for dialogue and discussion. It's nice to have faith in the spontaneity of the masses, but I find that waiting for the masses to respond only means a repetition of past failures. During 1969 and 1970 we witnessed the sudden growth of both the anti)war and women's movements, and the frustration and confusion we experienced the were due to greatly increased expectations on the part of too many people who literally expected instant solutions to long-standing problems. Luckily the women's movement has survived this onslaught of mass media publicity, since it has organized itself around women's needs (it is deeply rooted in women's lives), but the general movement has foundered for lack of direction and constructive activity.\\
At this time the anarchist movement is ineffective because it has yet to resolve basic problems of organization, communication and information flow, or learned how to settle internal disputes. Too much of our literature lacks fresh, creative or original thinking, and we are too little concerned with our internal problems, and too much of our association is marred by personality conflicts and petty disputes.\\
Much of classical anarchist thinking is visionary, inspirational, deeply ethical. Yet we need to develop a better understanding of how to actually create that revolution we desire; a critique of our daily lives. Radical feminism has come closest of any revolutionary movement to understanding the socialization process which stifles our ability to think and act freely, yet too many anarchist men (and women) continue to see feminism only withing the context of anti-sex (neo-puritans), role reversal or neurotic bitching, and are so threatened by the challenge to accepted behavior patterns and thinking they prefer to ignore or trivialize its meaning.\\
I am not convinced that anarchism works when I see a movement so loaded with hypocrites, moral cowards and intellectual snobs incapable of integrating their principles with their lives, of taking a principled position and of translating theory into action. I'm not asking us to get out the guns and ammunition (a loser's game, anyway), I'm asking us to quit hiding behind a wall of rhetoric and naivete and to work out some practical ways of resolving our problems, initiating meaningful activity and seriously creating a revolutionary movement.\\
Otherwise we can all stay home and watch the empire crumble on TV. Unless, of course, the energy crisis gets too close to home.

\chapter{Doing Anarchism Yourself}
\chapterauthor{Robert Shea}

Deciding to call oneself an anarchist is usually the end of a long process of eliminating other philosophies and the beginning of another long process of learning what anarchism really is all about. One new thing that has to be learned and is rarely fully appreciated is the role of the individual in anarchism. It's easier to give lip service to an abstract ideal of individualism than it is to act as if you actually believe that no one and no group and no principle is more important than you, yourself. Or more important than any other particular person. By this belief you commit yourself to no longer passing the buck to others. You declare your preference for a way of life in which you make all decisions and take all responsibility and expect to rely on your own thinking and your own efforts. This is a difficult ideal.\\
The anarchist movement in this country is very small and has difficulty making itself heard. This is due, not so much to any inherent implausibility in anarchist ideas, but rather to the problem that once people accept the ideas there seems to be no clear-cut direction to take. Groups come together and drift apart; publications appear and vanish; actions and demonstrations are staged and forgotten. Meanwhile, governments and state-supported economic systems seem to grow stronger. Sometimes it seems hopeless.\\
Here and there much is accomplished, often by one person or three or six. One group in one place will get its energies together and be printing and distributing literature, organizing, picketing, demonstrating, finding imaginative ways of getting the message across, setting an example for the rest of the movement. But in many places another picture is typical: Nobody knows what to do. Meetings are boring and seem purposeless. If new people come into the group they soon drop out for lack of promising projects to work on. Constructive suggestions or good ideas die on the vine because objections are raised or because no one follows through. It seems impossible to raise money. The group is divided by personality clashes. Publications die after an issue or two. The temptation is great to chuck it all and go to work for some local reform candidate or join some local Marxist group, just to have a feeling of accomplishing something.\\
Of course, the desire to accomplish something is itself a product of authoritarian thinking. Authoritarians are always measuring themselves by their numbers and the size of their accomplishments. During the first couple of million years of our species' existence on this planet we didn't accomplish much. Life for hunting and gathering people would seem relatively monotonous to a modern person. The really heavy demands for accomplishment must have originated with the domestication of man, with the invention of the state, when work and production quotas were set by priests and kinds and taxes were imposed. The hunting and gathering life would probably have been much more congenial to anarchists. The accomplishments of individuals acting on their own or in small groups need to be measured by different standards than those of disciplined mass organizations.\\
At the same time, it would be good if we could achieve more than we are doing now and if we could feel better about what we do achieve. Not for the sake of assuage guilt, not to feel that we're as good as various political movements, but simply to keep anarchism itself alive. Our main task today, as I see it, is not saving the world, but the humbler goal of keeping anarchism going, so that those who think as we do will be more numerous and more influential in the next generation than they are in this. If the race can manage to survive without our help for a generation or two more, there may come a time when we actually will have the strength to halt the doomward march of civilization and turn it around.\\
The sense of futility that afflicts some anarchists is due, I think, to each person's expecting others to come up with ideas, to waiting for someone else to make the first move, to seeking group approval before launching a project, to expecting others to take some of the responsibility and help with some of the work, to limiting one's own role in the group to the performance of some specialized function. In some people, this excessive reliance on others extends even to an unwillingness to define for oneself the aims and strategy of the movement. The result of this dependency is that nothing gets done unless some obvious project of fortuitous opportunity for action galvanizes everybody at once. Otherwise people blame one another for the group's failure to act constructively and eventually drift apart. Some turn to authoritarian movements of the left and right because that's where the action seems to be; others go into more conventional politics; still others subside into quietism. Or leaders may arise in the group and take charge. This can help for a while, especially if the leader is tactful and the group understands the practical need for temporary leadership. But in an anarchist group sooner or later a faction is likely to arise that will reject the leader just because the mere exercise of leadership, no matter how informal and non-coercive, is thought to be inimical to anarchism. In any event the emergence of a leader can be disruptive and, while postponing disintegration for a time, will lead to it in the long run. And the breakup of the group, unfortunately, often means the abandonment of anarchism by the members of the group as a philosophy that doesn't work. I think the real reason for these developments is not that anarchism doesn't work, but that it often isn't fully thought through and applied. What's needed are distinctly anarchist ways of getting things done.\\
There is an authoritarian pattern of group action that is pretty much universal regardless of the group's purpose or political-economic philosophy. Primary is the assumption that the group needs a leader and that very few people are capable of leadership. The responsibility for setting aims, coming up with ideas for action and assign tasks then belongs to the leader. Followers do not expect themselves to work as hard as the leaders, or to have ideas or display initiative. Among authoritarians the division of labor produces the impression that most work is unpleasant and undesirable, resulting in a demand that the drudgery be distributed equally among members of the drudging class. Everyone is expected to contribute an equal amount of labor to the group's well-being; there are to be no slackers. One of the tasks of the leader is to keep everybody busy, get everybody to obey orders. At some point in humanity's past it became customary to use force to control people. This was the origin of the state. As Jacques Ellul puts it in \emph{The Political Illusion}, `` To say that the state should not employ force is simply to say that there should be no state."\\
In authoritarian thinking there is an invisible line which people cross when they become members of the group. They have in their own minds, and in the minds of other members, surrendered a certain amount of personal sovereignty. The welfare of the group becomes more important than individual self-interest. It is this sacrifice of autonomy that supposedly gives leaders the right to discipline and punish.\\
Democratic organizations don't differ from this pattern, and such groups are also authoritarian. Although decisions are made and leaders appointed by vote, the leader-follower structure is the same, as is the claim of the supremacy of the group's will over the individual will. Once a decision is made by vote all members are expected to support it whether or not they agree with it. As has often been said, democracy does not do away with tyrants, it merely makes the majority a tyrant.\\
What happens when people reject authoritarianism? What they are rejecting, essentially, is the use of force to coerce the unwilling. This is the essence of government. In order to justify its use of force, government claims that the welfare of society has a higher moral claim than the individual will. To reject that claim means that you rate individual autonomy as the higher value and hold that a group or its leaders are never justified in forcing a recalcitrant member into line. (Of course, it is possible to take the position that the group's welfare is more important than the individual's welfare \emph{but} that the group has no right to coerce the individual. But I can't see on what basis the group could be so limited.)\\
Anti-authoritarian groups generally have no constitutions, no leaders, no formal procedures for reaching decisions. They are usually open to all comers and neither officially induct nor expel members. So membership and its obligations tend to be self-defined, or at least a matter for debate. The clear line which distinguishes authoritarian group members from non-members is missing. In many cases, however, these groups, though they have rid themselves of the trappings of authoritarianism, have retained authoritarian habits of thinking. Also, they have jettisoned authoritarian methods of getting things done, but they have not developed anarchist ways of doing things. For example, the idea that there are few qualified leaders and many followers often induces people to wait unconsciously for a leader, to expect strength and guidance from an external source, even though they had explicitly rejected the principle of leadership. Then there are those who spend all their time denouncing anyone else's initiative as an attempt to lead. Still other people are inhibited from showing initiative, fearing that they harbor within themselves impulses toward leadership.\\
Anarchists also carry authoritarian notions of the division of labor into their groups. This mainly takes the form of those who are productive resenting the non-contribution of less active members. One person may be working 16 hours a day while ten people who claim to be part of the group appear to do nothing. Since discipline is considered un-anarchistic, the usual way of coping with this imbalance is to harangue the less productive members, warning of dire consequences if everybody doesn't get behind the program. Haranguing, unlike coercion, is consistent with anarchist principles. Sometimes, if the energetic members get fed up with the invincible apathy of the others they may quit and let the group founder. Rarely do the hard workers expect gratitude, but the unfair criticism they are sometimes subjected to --- charges of trying to be leaders, gurus or superstars --- can sometimes be the last straw. Too often, instead of manifesting Kropotkin's ideal of mutual aid, anarchist groups destroy themselves through mutual recrimination.\\
Another hangover of authoritarian thinking in anarchist groups is the tendency to treat personality clashes and factional fights as catastrophic. Intramural conflict is as common as colds in winter in all groups, authoritarian and anarchist, in this culture. Among authoritarians such clashes frequently lead to splintering or to the annihilation or expulsion of one faction. But most authoritarians treat serious conflict as being intolerable. It is --- if one wants to achieve the machinelike unity and discipline that is the authoritarian ideal. But conflict should not be intolerable for anarchists. In fact, people are attracted to anarchism because they feel they should be able to fight in peace.\\
Anarchists also frequently retain the authoritarian idea that there must be formal approval for a project. Anyone with an idea in mind feels a need to get a consensus before acting. Frequently since majority rule is not accepted as a principle, nothing will be done unless there is unanimous approval. I have seen one person's objection stymie action on a perfectly sound idea because the rest didn't want to go ahead without total approval.\\
Right now, it seems to me, there is a flourishing mystique of excessive reverence for groups, of deification of the collective will. Team spirit is expected to replace leadership. In part this is the result of a feeling that the movement of the 60s was conned and ripped off by some of its leaders and spokesmen. In part it comes from a general revulsion from patriarchal authority spurred by the women's liberation movement. In part it seems to be an identification of the evils of capitalism with individualism. This belief in the superiority of the communal mind goes a long way back; in post-Revolutionary Russia there were experiments with conductorless orchestras (but not, so far as I know, with captainless ships). This attitude is understandable and healthy, but it assumes that there are only two ways of getting things done, authority or consensus. Carried to excess, this supposed respect for the group's opinion can mask laziness, apathy, timidity, inhibition, obstructiveness, unwillingness to take responsibility. Everyone sits paralyzed, waiting not for a leader but for a mythical entity to take charge, the group mind. And because there's no such thing as a group mind, the energizing impulse never comes. This worship of the communal spirit is authoritarianism without authority, all followers and no leaders.\\
It is simply very difficult to take the individual seriously. Over 10,000 years of authoritarian programming work against it. The I.W.W. anthem refers disparagingly to ``the feeble strength of one." Individual desires are small, mean, petty, selfish --- ``mere." Almost all moral systems are based on the belief that to be binding a moral code must be authoritative for everyone. An ethical theory based on the idea that each person's morals are binding only for him or her is dismissed as being bases on sentiment, whim, subjectivism. Much traditional U.S. rhetoric, like much anarchist ideology, pays lip service to individualism, but authoritarianism is rampant in this country. Ayn Rand's Objectivism, supposedly the individualist ideology \emph{par excellence}, rigorously throws out heretics and dissenters. In the mysticism that has captured the imaginations of many bright people, ego is a dirty word.\\
And yet, individuals are really all the anarchist movement had to work with. A big step toward learning to be an anarchist is to look for and follow the light withing. Only when anarchists start taking their own individuality seriously will the movement get off the ground.\\
If a meeting seems boring and purposeless, why blame the group and wait for someone else to do something? Require yourself to produce positive proposals and speak out. Don't wait to be told what to do; think of things for yourself to do and announce that this is what you are going to do. If you need help, ask for it. If you get help, you have group united by a common project, rather than a collection of people bound only by an amorphous commitment to anarchism and a vague agreement about the terrible state of the world. If no one wants to help, then you, the person who had the idea, should try if possible to carry it out alone. Anarchists have to learn self-discipline and self-sufficiency.\\
Don't expect the anarchist movement to match the performance of authoritarian political organizations. From the point of view of real freedom, the marvels of capitalism or communism mean little. The source of real creativity is the unfettered individual mind, and its accomplishments permeate a culture subtly, without the help of marching masses, disciplined parties, overwhelming technology or armed force. Let us be content to accomplish things on a small, local, human scale --- a Cro-Magnon scale --- believing that such achievements take deeper root and have more value in the long run.\\
Those who have learned to be thoroughgoing anarchists do their work because they enjoy it. They get pleasure from the excitement, the commitment, the hope for the future. They are full of ideas of their own and not terribly concerned about what others think of them. If they think something is worth doing, they'll do it, without taking an overt or covert vote. Therefore they don't get discouraged if others don't help. Some people simply have an abundance of energy and dedication and some do not. Just expend as much as you have, without looking over your shoulder to see how much others are doing.\\
Not every anarchist can be a bundle of energy and a model of total commitment. Since they consider themselves more important than any group they are part of, real anarchists may withdraw from projects that don't interest them. The group has no claim on anyone; it exists to satisfy the needs of individual members.\\
Groups do not do things. Everything begins with an individual impulse. Even something that looks like teamwork breaks down, on close examination, to the resultant of a number of individual contributions. It all starts, not with a group soul, but with a single person. Each anarchist has the right and the responsibility to define the total picture of the movement, what its principles are, what its aims are, where it is going, what its strategy and tactics should be. No one should sit back and let a few ideologists do all the thinking about the big picture.\\
We are all victims of an authoritarian mind-set that dates back, at least, to the neolithic era. The anarchist movement, little more that a century old, represents a beginning effort by some members of our species to erase that programming and try to think about human problems in a new way. This new thinking and doing, whatever it may become, will not originate with leaders or groups. It will come from individuals, from the voice and the light within.

\chapter[The Necessity for Pacifism]{Do an-archists know how to read? or \\The Necessity for Pacifism}
\chapterauthor{Joffre Stewart}
\fancyfoot[L]{Spring, 1975}

\section*{1.}
An-archism begins in non-violence. The logic that tells us that an-archism is rooted in non-violence parallels the chronologic which makes Lao-Tzu the oldest an-archist of record with his Tao Te Ching dated as far back as 2500 years ago. An-archism begins in China but this is not necessarily an argument for cultural diffusion. Many people discover the idea of autonomy for themselves which is almost the only way and then relate to (the sociology of) an-archism when it is brought to their attention.\\
Zeno of Citium, founder of the Stoic school in Greek filosofy, was an an-archist, probably not the first, because he learnt from the Cynics. The filosofical cynicism of the ancient Greeks was apparently a more agitational and social change thing than what is meant by ``cynicism" today. Today it connotes passivity and resignation to things as they are, a wisdom that is no harnessed to doing anything good. Thus E. Howard Hunt cd be a cynic as well as might Lenny Bruce, who, for all I know, carried a draft card, paid taxes, and engaged lawyers: he was not an activist. What these Greeks had to say tended to be negative to slavery \& violence and the civilization that was elaborated on them. You cd not distill a Red Army Fraction from their theories.\\
The next important an-archist in the history of nonviolence (nonviolence seems to have started as an-archism) is Jesus. Mark 10: 42-43 defines an-archism:
\blockquote{$^{42}$ Jesus called them to him, and saith unto them, Ye know that they which are accounted to \emph{rule} over the Gentiles, exercise \emph{lordship} over them; and their great ones exercise \emph{authority} upon them.\\
$^{43}$ But so shall it not be among you:}
As long as an-archists are unable to see that Jesus is one of us, an-archism is in trouble, and, considering the stubbornness of resistance to perceiving this, B I G $\;$ T R O U B L E. Jesus is the most widely disseminated teacher in the world, the most respected (if only superficially) in the Western world, and perhaps second only to Marx in the world as a whole.\\
The failure to dig Jesus as one of us means that we don't know how to read.\\
The failure to appreciate Jesus as one of us means that we underestimate a latent mass potential for an-archy. It means we are throwing away tremendous ``capital." This failure means that we have not penetrated and overcome the Establishment brainwash \& interpretation that ignores, subverts and perverts Jesus' ANTI-State, ANTI-Authority orientation. And this failure to overcome, and therefore to incorporate, Jesus in our history with Bakunin, Joan Baez, John Lennon, Anne Hutchinson, etc., means that we ourselves are not only the victims of this status quo propaganda, but that we cooperate in our victimization. It means that we are suckers for ruling class propaganda. If we are suckers for ruling class propaganda, do we have the strength of mind to prevail in the first place? And if we did gain without assimilating Jesus to our history and comradeship, then do we understand enuf about brainwash, propaganda, and ``co-option" to keep and advantage once gained?\\
The acceptance of Jesus is a test for the survival of the an-archist movement.\\
Now, Jesus, the above indicated Greeks, and Lao-Tzu were an-archists (an-archism \& non-violence being identical) because the polities in which they lived were so violent. An-archism was developed as as solution to this problem, \emph{WAR}, not as a justification for going to war. If those ancients had nonviolent reasons for becoming an-archist, then the radioactivity in the bones of so many of us shd be reason enuf to make us, if not super-pacifists, then at least pacifists, which is the beginning of adequacy in the post-Hiroshima situation. And one is certainly not adequate to an \emph{an-archist} solution if, like Tyrone Walls, you are uncertain how many megatons to deploy after you let the `ruling class' maneuver you into that situation.\\
Notice, that in the above, I use history to b r o a d e n consciousness. The Marxist uses history to \emph{narrow} consciousness, e.g., defining Imperialism in terms of \emph{capitalism} (you'd think Imperialism was born yesterday). This \emph{narrowness} tends to fanaticize consciousness, tends to blind them with righteousness to their own use of Power, and it leaves them fighting among themselves abut capitalist-ANTIcapitalist USSR, China, etc.

\section*{2.}
Since an-archism originates, logically, in the negation of coercion-violence, and chronologically, as a solution to war-slavery, the class analysis is not essential to a viable theory and practice, War makes slaves. Almost every black into ameriKKKa came as a prisoner of war. (1) Class does not establish conquest (2) Conquest establishes class. (3) Class does not establish itself. (4) Economics (labor + `exchange') is not coercion (5) Economics + Authority  = economic coercion (``class") (6) Taxation is economic (7) Taxation is economic coercion (8) Taxation is economic coercion but not ``class" (rent interest profit). Marxists cannot deal with this --- (6), (7) --- because they \emph{WANT} to be guilty of it (9) Law perpetuates conquest (10) Law perpetuates ``class", etc. (If there is any oversimplification in (1)--(10), the truth is in the oversimplification, not in obfuscation.) Marxists in particular are especially ill-trained to understand (1)--(10). Their economic analysis is no more than an excuse to get into politics, therefore an excuse for making war, for gaining State power, and therefore for shooting at each other like Marxist Russia and Marxist China who are into proving once more that \emph{war is in the nature of the State}. But there is no better blinder than Marxist economic analysis to prevent Marxists from seeing that they are demonstrating the an-archist maxim that \emph{war is the nature of the State}. Russian and Chinese Marxists won't even be able to see it in the maximum illumination of mutual exchange of a million megatons. The trouble with this is that we all die for their blindness.\\
It is useless to point out that Marx mentioned primitive accumulation or agreed in part or in essence with (1)--(10) above, because the essence of \emph{Marxism} does not consist in what is reasonable in Marx's economic writings which are irrelevant. \emph{Das Kapital} is the particular economic analysis which is considered the base of the Marxist system. There is nothing in \emph{Das Kapital} that suggests parliamentary procedure, yet this is the most unquestioned, unshakable foundation of Marxist practice. It may be that there is \emph{nothing} in \emph{any} of Marx's writing regarding parliamentary procedure; it may only be assumed in some letters where the task, for example, is to suppress FREEDOM NOW (an-archism). There is nothing in \emph{Das Kapital} that suggests resort to State power, yet Marxists can sooner believe in virgin birth, that not using the State. There is nothing in \emph{Das Kapital} that persuades the reader to accept obligatory servitude, yet Marxists, whom one might at first assume to be libertarian because they are fighting against the Selective Service System, surprisingly, but not infrequently support the principle of conscription themselves. Something in \emph{Das Kapital}, regarding the \emph{corvee}, might suggest that Marx wd be unsympathetic to coercion of labor but Marxists cannot dream of NOT paying Nixon's salary. \emph{Das Kapital} is not prejudiced against pacifism, but Marxists think pacifism ridiculous.\\
Marxism always means what is hostile to FREEDOM NOW. Marxism always means what is hostile to FREEDOM NOW regardless of what may or may not be quoted from the young of old Marx. The political monster, Marxism, ultimately derives from the spirit and practice of Marx, which, as the spirit and practice of Authority, went to war against FREEDOM NOW wherever it crossed his path. In Paris he opposed Proudhon. In Berlin he opposed Stirner. In the context of the First International, he fought Bakunin. And in the conflict with anti-Authority people, Marx rigidified his theory in the more nonhuman, deterministic, nonlibertarian (i.e. ``scientific") directions. The monster became more monstrous as it developed and survived in conflict with anti-Authority forces.\\
(A wishy-washy an-archism cannot deal with this).\\
One readon for this (this outcome) is that we live in the wake and backwash of wars. I grew up, at least in part, in the patriotic backwash of The Big I and then there was The Big II to saturate movies and TV and date the times.We do not date the times by coronations and not really by Jesus since he was absolutely opposed to lordship \& domination, but by orgies of patriotic bloodletting; Post (-World) War I, Post (-World) War II, World War III (pre-Armageddon)... Anti-Establishment authoritarians (and others) of the 19th Century looked to the French Revolution as a model and we live in the backwash of the Russian Revolution which is used to define new terms for domination and spin off new models for violence. And before the new terms of self-deception are established we may see the old ones applied to the new model: the Stalinist system, for example, analogized as ``Bonapartist". With a pacifist consciousness reinforcing out an-archist one (how did they become separate?) we can really dig in and attack smash destroy the assumptions of both the patriotic and the labor-chauvinist models for social violence.\\
Another reason why Marxists are always against FREEDOM NOW may be owing to ethnic origin. Carl Landauer is a socialist, not an an-archist, but in his \emph{European Socialism} he has this to say:
\blockquote{``He [Bakunin] considered the Germans and the Jews the two races with the most authoritarian spirit, and since he regarded this spirit as his deadly enemy, his hatred of the German Jew (i.e. K. Marx), who stood for a rather centralized form of socialism, was very intense, at least in the later phases of their relationship."
\par\begin{flushright} Page 128. University of California Press. Berkeley \& Los Angeles. 1959. \end{flushright}
}
We shd keep in mind that on the official German record, the atheist Marx (author: \emph{A World Without Jews}) was of Lutheran origin, not Jewish, otherwise he cd not have qualified to become a Doktor. Some U.S. Marxist sects may be 80\% Jewish. The eastern European character of U.S. Jewry may mean that the authoritarian practice of Marxism may have been influenced by the factious, discriminating method of theological disputation of that ghetto Jew who replaced the Talmud (or whatever) with the writings of the Herr Doktor professing Karl. Hence, sectarianism as we know it.\\
One cd go on forever both detailing and conjecturing what is wrong with Marxism and Marxists. It is impossible to be too devastating in attacks on Marxism, especially since Marxism orients what is called ``the movement", including the so-called ``anti-war movement". However, attacks done ineptly can be counter-productive. And attacks on Marxism are not attacks on Marxists --- our relations to them shd be pacifistic, of course.

\section*{3.}
If the unity of an-archism and nonviolence (anarcho-pacifism) were as obvious, in the above, as I wd hope it were, then the following questions wd not be asked, but they are asked:
\begin{enumerate}
\item Can people dominated by colonialism get independence by pacifistic means?
\item Isn't it better to make a few gains by violent or parliamentary methods rather than attempt the nonviolence that wd take longer?
\item In time of danger isn't it necessary to defend workers' rights against attacks from the upper class?
\end{enumerate}
1.-2. The first two questions can be answered with one expression: Mahatma Gandhi. The subcontinent did separate from Britain by pacifistic means, means that owe something to the an-archism of Tolstoy and the libertarian individualism of Thoreau. But not enuf, because separation from Britain was achieved by the political form of \emph{independence} rather than in the nonpolitical-ANTIpolitical form of an-archy. The only proper goals for nonviolent action are those consistent with STATELESSNESS (FREEDOM NOW!). It is the grossest perversion of nonviolence to use it to reorganize the forms of domination, and this was evident on the subcontinent long before Gandhi (a liberated person?) exploded the nuclear device that can be intercontinental in 1979. India split from Britain by splitting in 2, and 8 million died in a matter of months, not years like it took to achieve the same gross in WWI. Kashmir War. Two with China. Then Pakistan split in 2 in the bloody affair of Bangla Desh. This is quite counter-productive to the anti-archic aspiration because this aspiration is to \emph{reduce} the number of sovranties on Terra to zero, not to increase them from zero to 3, as in the subcontinental case. Just as an-archism in the unself-contradicting sense, is nonviolent, so nonviolence, to be consistent with itself must be an-archist. Non-violence is not a method of multiplying and enlarging armed forces.\\
Peoples dominated by colonialism \emph{can} get independence by pacifistic means but is the human situation improved now, now that they rape \& slaughter each other by lakhs, millions, crores? Is it worth a Gandhi-made H-BOMB exploding over your head? Is it worth an India-China nuclear war that omnicides the human speecies?\\
The Indian independence struggle, credited, properly, to Gandhian nonviolence (tho there was mutiny in the Indian navy and perhaps incipient guerrilla warfare) took about 27 years and cost 8,000 lives. It is not inconceivable that armed struggle cd have cost 80,000 lives or 800,000. The Indo-Chinese struggle has been on and off for much longer than a generation, at least 800,000 died \emph{during the U.S. intervention alone}, an-anarchists cannot expect security from Marxist sectarianism in North Viet Nam, both Nams have conscription, and Saigon is capital of the largest computerized police State in the world, thanks to the American taxpayer. Before the agreements were signed to start the cease-fire war, it was revealed that the U.S. had plans to stay in Nam 10 more years, and it may well be that the Council on Foreign Relations has no intention ever of abandoning the ameriKKKan presence in Viet Nam. Nonviolence does not necessarily take longer than armed struggle, it costs less in economic drain and in lives, and the gains of exchanging one form of domination for the other are about as illusory in one case as the other.\\
\emph{Parliamentary methods?} Such methods separated ``Nigeria" from Britain, and the war that followed, 7 years later, to reduce the only \emph{legitimate} government in Nigeria, that of Biafra, (was preceded by racist electoralizing) was conducted more intensely than in Viet Nam, and was more frightful, if possible, than the fission that produced Bangla-Desh. Is this what is meant by \emph{``a fes gains by... parliamentary methods"}?\\
Marxists support national liberation movements because Marxists are always against FREEDOM NOW, and the Pearl of the Antilles makes their \emph{venceremos} example. Armed struggle in Cuba was of shorter duration if you count from Moncada, than in the India case --- and the whole world nearly paid for the gains on October 22, 1962. A choice for nonviolent struggle cannot be regarded as as mistake even if nonviolent resistance were still unconcluded in Cuba, about 20 years later. However, the struggle wd be pointless if `making a few gains' left the principle of sovranty intact, or exchanged subordination to the U.S. State Dept. for subordination to the Kremlin, as in the armed struggle case.\\
Peoples dominated by colonialism shd not have to do, proportionately, as much as they do. We are in the belly of the monster. Surely more can be done than throwing Dow Chemical files into the streets, if only more of it. Most of those who term themselves \emph{anti-imperialists} are into ``organizing", not action, into education, too little of which is education for \emph{DIRECT ACTION}, and into making prayer to that which delivers domination instead of wrecking the delivery systems of domination.\\
An-archists are much more lacking than Marxists in what concerns giving specific attention to anti-colonialism, so it is not to be wondered that Marxists get ahead in this area. When an-archists think of a transnational conference, they think of Europe. Not even Latin America is considered. Those an-archists who compromise themselves with passports shd think of having the next transnational congress in India, in the presence of the Sarvodaya movement which is (wishy-washily) an-archic. The Atheist-Who-Walked-with-Gandhi (a liberal) is another possible host for such a subcontinental gathering.\\
\emph{3. Emergency defense of workers' rights?} The best way to answer this is to point out that the upper class (or whatever) is in position to attack workers' rights, today, because workers and others in previous times made themselves content merely to defend those rights \emph{and did it in such a way as to postpone Utopia}. \emph{The State delivers domination} (not Wall Street --- the USSR has no Wall Street) and the pacifist attitude --- equally an an-archist attitude --- of TAX-REFUSAL NOW is always a defense of the workers. The an-archist case is one of prophylaxis: \emph{SMASH THE STATE} before it gets pregnant with a Hitler, Stalin, Franco or whatever. Socialist repression (NSDAP, CPUSSR) seems more efficient than much we've seen in modern times, tho the CIA sis no slouch (Indonesia, Chile). A `successful' defense of workers' rights by Leninist means, may be the worst thing that cd happen to workers.\\
CAVEAT: removing the State as a means of delivering domination may be too narrow an anti-Authority approach if offered as the unique solution to racism-sexism. This kind of caution is, of course, even more relevant to the kind of socialism that overemfasizes economic causation and defers direct action solutions for racism-sexism. That kind of thing, more than anything else, shd be recognized as a crypto-racismsexism.\\
Since an-archism is anti-Authority, it shd be easier for an-archists to have more flexibility than economic determinists and tangle with racism-sexism both as manifestations of Authority and violations of an-archic equality.\\
Nonviolence delivers more means for dealing with these problems than does armed struggle which is counter-productive (or ridiculous) if it can be applied at all.
\begin{center}
F R E E D O M $\;\,$ N O W
\end{center}

\chapter{Free Love, Sexism and All That}
\chapterauthor{Robert Anton Wilson}

\blockquote{I have said what I have said; I have not said what I have not said.
\par\begin{flushright} --- Count Alfred Korzybski \end{flushright}
}
\blockquote{Lie down on the floor and keep calm.
\par\begin{flushright} --- John Herbert Dillinger \end{flushright}
}

Anarchism derives from the Greek \emph{an-archos}, without rule, and thereby implies individual freedom. To a naive observer, this would seem to necessarily include sexual freedom; and, indeed, the majority of anarchists, until about 1968 e.v., believed that an anarchist society would be a sexually free society. Such pioneer libertarians as Josiah Warren, Ezra Heywood, Stephen Pearl Andrews, Emma Goldman, etc. vigorously joined the socialist-feminist Victoria Woodhull in defending the ideology called ``free love," i.e. sexual self-determination. The thought ``individual freedom includes sexual freedom" was so ubiquitous at the beginnings of modern radicalism, in fact, that it actually was espoused by Frederick Engels (although not, of course, by Karl Marx) and spread far beyond anarchism into the mainstream of revolutionary ideology.\\
In the past few years we have learned that this seemingly self-evident inclusion of sexual freedom withing personal freedom is not self-evident at all. Many believe that an anarchist society would not, should not and could not allow free love. Sexual self-determination, we are soberly assured, is a form of sexism, and sexism is tyranny; ergo, anarchist society must set limits on freedom just like any other society.\\
I propose to examine this argument and demonstrate its total falsity. The classic declaration of free love --- Victoria Woodhull's famous outburst, ``I have an inalienable, constitutional, and natural right to love whom I may, to love as long or as short a period as I can, to change that love every day if I please! And with that right neither you nor any law you can frame have any right to interfere" --- is, I will show, the only sexual philosophy consistent with anarchist theory.\\
There are nine forms of anti-free-love ideology currently espoused in anarchist publications; I will discuss them in ascending order, from the most amoeboid tropism-level up to the mental or philosophical level.\\
First there is the anti-free-love \emph{reflex}. This occurs in persons who have never learned to use the higher neurological circuits of the cortex; it exemplifies the classic reflex arc beloved in behaviorist theory. The phototropic moth flies straight into the torch, and dies; the conditioned dog in Dr. Pavlov's chamber of horrors drools when the dinner-bell rings; many contemporary radicals just as automatically froth at the mouth or burst into angry speech (glandular spasms defending ideological ``territories" based on old mammal reflexes) when ``free love" is mentioned. Often, there is no thought involved at all; such persons merely know that it is fashionable to have this reflex at this time.\\
The pernicious prevalence of this type of robotism is largely die to the influence of Franz Fanon, whose \emph{apologia} for revolutionary rage has been widely used and abused as a rationale for retaining into adult life the trigger-reflexes of human infants or even of pekinese dogs. This ``existentialist" politics ---
\begin{addmargin}{2cm}
When in danger or in doubt\\
Run in circles, scream and shout
\end{addmargin}
--- wishes to reduce human neurology to that of, say, the rabbit's fear-immobility mechanism. It is worth noting that this freeze-reflex works well enough when the rabbit is in tall grass or shrubs, where statue-like stillness may render him invisible to predators. When the rabbit blindly follows this reflex on the highway, he usually gets run over. Evolution has formed the higher nervous circuits as monitors or feedbacks, correcting this ``blind' robotism in the lower circuits.\\
True freedom, then, begins with neurological freedom --- detachment from the mechanistic emotional-glandular reactions. In this ``cortical delay" (as Korzybski called it) so-called ``free will" begins --- not the gift of the gods, as theologians imagine, but the result of self-work and self-reprogramming. Lacking such integral meta-programming we continue to repeat the old imprints of territorial rage-fear going back to the Carboniferous amphibians, all still coded in the midbrain and ready for operation whenever the higher circuits are deliberately bypassed.\\
The anti-free-love reflex, then, is not merely anti-logical but anti-neurological. It uses the oldest, most primitive part of the brain and turns off the more recent circuits of the past hundred million years, especially the cortical development of the hominid stock, since 100,000 B.C.E.\\
The second variety of anti-free-love ideology is based on \emph{argumentum ad hominem} and/or psychological ``analysis." Thus, when the free love issue was debated in the \emph{SRAF Bulletin}, the opponents of sexual self-determination seldom argued the theory itself but merely reiterated endlessly that exponents of free love are (a) sexists (b) male chauvinists (c) compulsive masturbators or (d) otherwise contemptible, i.e. pissy, shitty, nasty, etc. (This is not a caricature of their writings but an accurate report.)\\
This is not a substantive argument. It might be proved conceivably, that all SRAF writers who support free love are somehow vicious characters, but this would not itself demonstrate the falsity of their position. Similarly, if all Darwinians were shown to be addicted to bestiality, this would \emph{not} mean that evolution is false and we must accept \emph{Genesis}; if all modern geographers are boot-fetishists we need not then accept the Flat Earth theory; etc.\\
The third anti-free-love argument is the claim that discussion of the subject is itself invasive or coercive; that is, to debate the issue is to cause acute psychological damage to certain persons. The same objection may be raised against geology by a Protestant Fundamentalist, or against discussion of cancer in a house where somebody is dying of that disease. This is most certainly a problem of tact, decorum and sensibility; but is it a legitimate bar against debate at all times in all places? I doubt it. Those who cannot enter the arena of political debate without being ``hurt" by the first manifestation of conflicting ideas certainly should leave that arena, but they have no real justification in demanding that the arena should be closed down.\\
On a fourth level of intelligence is the anti-free-love \emph{syllogism}. This is based on observation of or experience with various bounders, scoundrels, crumb-bums, etc. --- i.e. exploitative or invasive persons --- who have used (or misused) free love as a rationalization for their predations. The syllogism takes the form:\\
\indent Step 1: X, who is in favor of free love, is a bounder, scoundrel, crumb-bum, etc.\\
\indent Step 2: Y is also in favor of free love.\\
\indent Step 3: Therefore Y is also a bounder, scoundrel, crumb-bum, etc.\\
This is one of the modes of invalid syllogism, as Aristotle knew 2500 years ago; semanticists call it the fallacy of uncritical inference. The weakness is revealed by keeping the form and substituting terms, e.g. Step 1: X likes Jello and is an anti-Semite; Step 2: Y also likes Jello; Step 3: Therefore, Y is also an anti-Semite; ir Step 1: X smokes grass and blows great horn; Step 2: Y also smokes grass; Step 3: Therefore, Y also blows great horn. This is not thinking but revery.\\
On a fifth level, anti-free-love ideology takes the forl of asserting that freedom allows something reprehensible to happen. In the old days, this was supposed to be homosexuality, nowadays, it is supposed to be heterosexuality. Since nobody in anarchist circles would dare to raise the first claim at this time, I will deal only with the second; the two are functionally identical. If we are to exclude heterosexuality on the basis that FADA (Faggot and Dyke Anarchists, New York) or some other Infallible Authority informs us that it is indecent or neurotic or somehow immortal, etc. we have violated the very basis of libertarian thought. Others are then free to demand the suppression of homosexuality on the same grounds; although this sounds incredible today, such anti-Gay bias was once fairly widespread and might be revived at any time. In short, the fact that heterosexuality is currently considered evil or shameful is not in itself grounds for suppression.\\
Free love justifies voluntary, non-coercive homosexuality and voluntary, non-coercive heterosexuality. It cannot justify invasive homosexuality or invasive heterosexuality, e.g. rape or fraudulent seduction. And it cannot justify any attempt to abolish either homosexuality or heterosexuality.\\
A sixth form of anti-free-love ideology takes the form of historical analysis, or, as Karl Popper calls it, \emph{historicism}. This is the anti-evolutionary assertion that the origin of a thing is its ``nature" from which it can never diverge. Thus, if Relativity is of Jewish origin (Einstein being of Hebraic ancestry, although an agnostic), it is therefore ``Jewish physics" and thus untrue; this was a popular line with the Nazis once. Similarly, many anarchists believe that the historical evidence that the State originated in conquest is enough to discredit it; but it is \emph{not}. One must still prove that the State \emph{remains} an exploitative institution (which isn't hard to show, actually).\\
Anti-free-love ideology, similarly, is on unsafe grounds in claiming that free love was invented by men and is therefore a male plot against women. The reply is: 1. Origin is not essence or we would all still be fish. 2. The assertion is factually false; early free-love advocates Victoria Woodhull, Fanny Ward and Emma Goldman were \emph{not} men in drag. 3. Freedom is not a plot against anybody.\\
A seventh form of anti-free-love ideology asserts that sex is not itself important. This may or may not be true (\emph{degustibus non disputandam}), but it is irrelevant anyway. Free love means that a person may choose whatever sexual pattern seems preferable, so long as others are not harmed; this includes anything from homosexuality, heterosexuality, masturbation, triolism, orgies, etc. for those who think sex \emph{is} important, toe celibacy for those who think sex is \emph{not} important. Proclaiming this right is as vital to the celibates as it is to the swingers, if we are ever to have a libertarian society.\\
The eighth protest against free love is based on the claim taht \emph{sexual freedom} (as distinguished from sex itself) is too unimportant to be worthy of debate. While this school of thought does not explicitly reject freedom, it argues that sexual self-determination is merely ``individualistic" or only ``bourgeois" or some way foul and contemptible. This seems to me a total misunderstanding of neurology and psychology.\\
\emph{A person who can be sexually coerced has no identity.} The sexual function is so intimate, so biologically deep and interconnected with kinesic and proxemic self-definition, so ``touchy" and delicate, that when sexual self-determination is surrendered there is little selfhood left to fight any other battle. The woman who can be seduced by any scoundrel who comes down the pike; the devotee of a supernatural religion who has given up sex entirely at the behest of shaman or priest; the rapist with no ability to postpone gratification --- these are all strangely zombie-like types, schizoid and sometimes outright schizophrenic. If you do not ``own" your genitals --- if the Church owns them, or any stronger person in the room owns them, or some ``irresistible impulse" owns them --- you do not possess anything like self-identity.\\
The demand for sexual self-determination is as basic as one's own skin, one's nerves, one's bioplasm. By comparison, the right to own the product of one's labor, or to write and speak without censorship, are less important, because external. Sex goes right to the core, inside the ganglia, down to the cellular level. If ``they" (society, the State, whoever) have you ``by the short hairs," they \emph{really} have you. The battle for self-ownership begins right here; and if one surrenders on this level, there is hardly any spirit left to fight for the externals of freedom. Every tyrant knows this intuitively, and sets taboos on the sexuality of each people he hopes to rule.\\
The ninth argument against free love is based on the allegation that all men are scoundrels. (This is a variation on the syllogism discussed in point four and/or the \emph{argumentum ad hominem} of point two.) There are several answers, as follows:
\begin{enumerate}
\item "Allness" statements are semantically invalid. The lengthy, and somewhat technical, mathematical proof is given in Korzybski's \emph{Science and Sanity}; but, briefly, the arguments offered against anti-Semitism or anti-black prejudice in any liberal high school civics text apply here also.
\item Empirically, the equation \emph{all $m = K(s)$}, all men are in the class of scoundrels, is not accepted by \emph{all} humans (certainly not by men), nor even by \emph{all} female humans, nor even by \emph{all} feminists. The attempt to set policy on the basis of such a claim is, then, an endeavor by a minority to enforce their own phobias on the majority, as in the old puritanism, all forms of institutionalized racism, or tyranny in general.
\item Even if all men were proved to be scoundrels, this is not a \emph{refutation in principle} of free love, but merely a point of prudence or policy. That is, the principle of freedom would still stand but intelligent women would avoid make contact; they would still have the liberty, under anarchist theory, of sexual self-determination in choosing between celibacy, lesbianism, masturbation, bestiality, etc., just as the ``scoundrelly" men would have the liberty of choosing between similar alternatives. The liberty to decide remains a firm anarchist principle, even if wisdom strongly decrees that certain choices shoud be avoided. Cf. heroin of guns.
\end{enumerate}
As a sub-case of argument nine, it may be asserted that under the present patriarchal conditions, any sex between male and female gives all the advantage to the male, encourages him to be fraudulent or dishonest, discriminates against the woman, and generally leads to an ever-closer approximation of making ``all men are scoundrels" become, in fact, true. This ignores the evidence that many persons of both sexes are able to resist the temptation of scoundrelism, however strong that temptation may be. (Cf. pacifists in our jails, Germans killed for helping Jews escape, honest businessmen, politicians who refuse bribes, etc.) But this, again, is a matter of prudence and judgment, not of libertarian principle. Men and women, even under these authoritarian conditions, retain the inalienable right to decide for themselves whether to choose homosexuality, heterosexuality, monosexuality, asexuality or whatever pattern suits them; and this includes the right to decide whom they will and will not trust.\\
There are other arguements against free love but they are explicitly non-anarchist (authoritarian) and do not need refutation in this place at this time. To an anarchist aware of anarchist theory the issue remains clear: Love is either free or coerced. The former represents anarchism and the latter represents tyranny.



\end{document}


















