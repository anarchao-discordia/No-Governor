\chapter{The Frying Pan}
\vspace{-1cm}
\emph{Reviews of Libertarian Publications Recently Received}\\

\noindent\textsc{Against the Wall.} A libertarian magazine. Volume 4 Number 5 is largely book reviews, including \emph{A Gang of Pecksniffs} by H.L. Mencken, \emph{Against Our Will} by Susan Brownmiller and \emph{The Air Force Mafia} by Peter N. James. This issue costs 50 cents, but future issues will be 75 cents. Against the Wall, P.O. Box 444, Westfield New Jersey 07091.\\

\noindent\textsc{Black Star.} An Anarchist Review. A publication of the Social Revolutionary Anarchist Federation. Continues to improve dramatically with each issue, as the continent-spanning collective that edits and publishes it gains in experience. For us, highlights of Issue Number 3 were ``Homo Economicus" by Glenn Meredith, on the way capitalism has conditioned people to feel artificial needs; ``Quality or Quantity" by Jersey, on anarchist agitation among the people right around you; and ``Towards an American Anarchism" by Irving Levitas on 19th-century grassroots religious movements whose ideas tended toward anarchism. There's much more (32 pages) to make the 25-cent cover price a bargain. Subs are six issues for \$3, \$10 for institutions, free to prisoners. Black Star, Box 92--246, Milwaukee, Wisconsin 53202.\\

\noindent\textsc{Equality.} A Libertarian Review. A one-sheet publication usually devoted to profiles of lesser-known anarchists, together with information on their published writings and on writings about them. As these keep coming out, a collection of \emph{Equality} portraits will become more and more useful. Volume I, Number 1 is devoted to Jan Waclaw Machajski, Polish-Russian social theorist and revolutionary. Number 2, Voltarine deClayre, American anarchist writer. Number 3, a very complete and up-to-date bibliography on Bakunin. Number 5, Rudolf Rocker, editor and labor organizer. Number 7, Robert Reitzel, 19th century German-American anarchist. From The Kropotkin Society, Post Office Box 2418, Evansville, Indiana 47714.\\

\noindent\textsc{Libero International.} Published in Japan, in just three issues this anarchist magazine has become the indispensable guide to the libertarian left in Asia. Issue number 3, November, 1975, includes a profile of the Hong Kong 70s Front, an anti-authoritarian socialist group; a continuation of the history of hte Chinese anarchist May 4th movement; the first installment of a history of the Korean anarchist movement; the story of an effort by Japanese farmers to resist eviction for an airport to be built on their land; much more. Published quarterly. \$3 for six issues, single copies 50 cents; send money orders or cash, not personal checks. Libero Internation c/o CIRA-Nippon SIC, C.P.O. Box 1065, Kobe, Japan 650--91.\\

\noindent\textsc{The Match!} Long-established tabloid newspaper out of Tucson, Arizona. Articles are tough-talking and frequently take out after others on the libertarian left as well as against anarchism's customary enemies. The April-May, 1976 issue, Volume 6 Number 11 is full of good reading: articles on the swine flue scare, gun control, individualism and despotism, materialism and vengeance as an anarchist principle. A column called ``Random Shots" by editor Fred Woodworth sets the lively polemical tone for the whole paper. Sample copy 15 cents, 12-issue subscription for \$3. The Match! P.O. Box 3488, Tucson, Arizona 85722.\\

\noindent\textsc{Open Road.} A stunningly handsome 32-page newspaper from a collective in Vancouver, British Columbia. Describes itself as ``designed to reflect the spectrum of international anarchist and anti-authoritarian Left activities and to provide reports and analysis of popular struggles and social problems. It is not the organ of a political organization." Name comes from Emma Goldman's original name for her magazine, which was eventually called \emph{Mother Earth}. First issue includes articles on Greepeace, resistance in Chile, protest against Trident submarines, the Symbionese Liberation Army, Kansas City Yippie convention, the Movement ethic, Holly Near, Martin Sostre, the American Indian Movement, the late Phil Ochs and women's labor organizing. The cover price is 60 cents, but the publishers say they have no subscription rates and depend on people's donations. It's very impressive and worth whatever you send them. The Open Road, Box 6135, Station G, Vancouver, B.C., Canada.\\

\noindent\textsc{Solidarity Newsletter.} A handsome, large-format four-page publication. News, reviews and letters for a libertarian-left audience. Number 12 presents a review of \emph{The Dispossessed} by Ursula K. Leguin, a science fiction novel dealing seriously and in detail with an anarchist society of the far future; an article on an attempt at tenant control in some New York City apartment buildings; an announcement that \emph{Solidarity Newsletter} may merge with a publication called \emph{Synthesis}. 15 cents a copy, the issues for \$1.50. Philadelphia Solidarity, Box 13011, Philadelphia, Pennsylvania 19101.\\

\noindent\textsc{Southern Libertarian Review.} A magazine of the libertarian right, which is for private enterprise, takes some inspiration from Ayn Rand and is against the government up to a difficult-to-determine point. The August 1976 issue, Volume 2, Number 11 includes ``The Canal Issue" by E. Scott Royce, ``A libertarian Nation?" about an enclave in Jamaica by Adam Starchild, ``William Jennings Bryan" by Robert Brakeman, and ``In Defense of the Non-Aggression Principle" by Jarret B. Wollstein. SLR, c/o E. Scott Royce, 1236 S. Taylor Street \# A, Arlington, Virginia 22204. 12 issues for \$6.\\

\noindent\textsc{Sweet Gherkins from the Dill Pickle Press.} The January 24, 1976 issue includes an analysis of fascism in George Wallace's presidential campaign, as well as brief excerpts from the works of Rabindranath Tagore, Stuart Chase, William J. Fishman, Donald Ogden Stuart, W.C. Fields and Paul Goodman. Two of the most interesting items in the May 24 issue were a statement by Charles S. Pierce on morality and one by Margaret Macdonald on political language. Subscription 10 issues for \$1, single copy 10 cents. 
