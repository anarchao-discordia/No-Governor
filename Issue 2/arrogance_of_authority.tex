\chapter{The Arrogance of Authority}
\chapterauthor{Jim Bumpas}

Does the title of this article make you think I am going to discuss ``authoritarians?" Or the snotty arrogance assumed by many Marxists once they have read a little Lenin or Mao? That is a bit trite for an anarchist theoretical discussion. I want to help develop some ideas upon the distortions produced in our own practice by what I call the ``arrogance of authority."\\
I believe it is a mistake for anarchists to ignore the problem of authority and the related problem of leadership within our own ranks. Likewise I believe it is a mistake to try to obscure the problem as Spanish comrades did in the 1930s by using euphemistic terms such as ``influential militant" instead of ``leader." I do not oppose the substitution of such euphemisms as long as we develop a clear analysis around their usage and as long as we do not allow them to cloud our practice with destructive distortions and to make clarity of analysis very difficult.\\
These distortion in the practice of our Spanish comrades have recently been analyzed in Vernon Richards' ``Lessons," and to a lesser degree one is made aware of some of them in Sam Dolgoff's book on the Spanish Collectives. The arrogance which I want to analyze here is not the arrogance which anarchists assume when they occupy the traditional seats of power in govt and other hierarchical bureaucracies. I do not believe anyone reading this will argue that an anarchist can get a bulldozer to fly if only an authoritarian can be removed from the driver's set and an anarchist installed.\\
Rather, I want to discuss the arrogance that an organization and its members allow to develop which permit any ``influential militant" or ``leader" to climb into any driver's seat and yet retain any former close ties to the anarchist movement. I think this arrogance is very similar to that arrogance exhibited by the Marxist who insists his understanding of Marxism-Leninism-Mao-Tse-tung thought provides the only possible theoretical underpinning of revolutionary action. If an anarchist organization has not already prevent this problem by clearly analyzing its elements, then perhaps ruthless criticism and even ridicule are the best weapons to use against those who would be leaders OVER the anarchist movement.\\
What about leaders? Are leaders destructive and incompatible with our perspectives? And leadership? Should we avoid ``influencing" others? If so, what about propaganda? Education? Exemplary action? Is the only alternative to leadership, and to being an ``avant garde" simply to withdraw into our anarchist communities and try to live consistently with our perspectives? Anarchist islands in an authoritarian sea? There are some rip-off artists ready to promise just such an escape. And there are thousands of people living in such communities who are gaining experience in collective, solidaric living.\\
But anarchism is a set of perspectives which aim towards social change. That is a fact. Therefore, both people and social conditions of a material nature must be altered, changed, transformed, revolutionized. And people must become convinced of the necessity and desirability of such change, whether by experience, logic, insight or other means. The social action required to spread this experience, logic and insight throughout society involves the following process which I describe as ``leadership":
\begin{enumerate}
\item the development of class analysis;
\item the clear articulation of the destructive effects of class society;
\item the identification of both structural and human symptoms of these destructive effects; and
\item the selection of relevant and effective projects which aid us in any of the above 3 steps, or which accomplish direct social change in themselves.
\end{enumerate}
Obviously, if everyone in society followed these steps, or similar steps, revolutionary leadership would not be necessary. Or any leadership for that matter. For revolutionary leadership to be most effective, revolutionary organization is required. In spite of what I have just said, I believe we can and must avoid any situation where we refer anyone to our ``leader" (singular or plural) when asked, ``Take me to your leader." Our organizational ideal will best conform to our perspectives when we put into practice the old Wobbly slogan, "We are all leaders." If we have achieved this in our own organizations, then we might be able to develop the experience and perspectives we need to achieve the same in society as a whole.\\
This is really the crux of the differentiation between anarchists and our organizations as opposed to authoritarian, hierarchical organizations. Anything which develops the initiative, self-reliance and self-confidence of anarchists and of people in general reduces the present popular dependence upon authority. The tendency of our activity is and should be to make this dependence disappear. We are not like a Marxist-Leninist group trying to shift dependencies from ``them" to ``us" so that we may step up into the driver's seat of that bulldozer. We do not want to develop authorities on anarchism, or among anarchists.\\
Nevertheless, many anarchists appear to exhibit behavior which is conditioned by the authoritarian dependencies and which serve to perpetuate them, albeit laced with anarchist rhetoric. I refer the reader to the Bulletin of the Social-Revolutionary Anarchist Federation for both the evidence of anarchists exhibiting authoritarian behavior and manner of thinking as well as for the efforts of anarchists to combat the effects of authoritarian conditioning and to help one another develop out of the conditioning into more free modes.\\
Some anarchists assert that one is either an anarchist or one isn't. They assert that theoretical discussion, patient explanation and mutual exchange of ideas are both unnecessary and wasteful bullshit. This attitude is not limited only to those who want to lead us right to the barricades for a fight to the death against the force of reaction. It is also expressed by those who want to lead us into exemplary action of other types, such as developing alternate communities or media action for its shock value. Or maybe just propaganda is the object. In any case, this attitude exemplifies an incredible arrogance.\\
On the one hand, they appear to say that an anarchist appears on earth by some mystical process in a perfect form that discussion and theoretical perspectives cannot improve. The only task of anarchists is to take ``the word" to the people and internal education of ourselves is wasted activity when there are so many people out there, thirsting for our message. This arrogance assumes that anarchists need no further education once we decide that the word ``anarchism" best describes our social perspectives, and that we can educate people to accept something closer to our vision of a free society even if we cannot educate them to be anarchists. This is very like the Christian authoritarian evangelist who ``knows Christ" and you cannot understand unless you also know ``HIM".\\
Other anarchists who exhibit a similar arrogance believe that clear development of our perspectives and the words we use to communicate our perspectives are unimportant. They have internalized the Madison Avenue dictum that words can be made to mean what we want them to mean. In the resulting confusion, they believe people will pick up anarchism off the shelves as a result of hyping the product name and associating it with any and every popular concept which has permeated our popular culture. To concretize this analysis, I need only list the words ``hippie," ``free market," ``Woodstock Nation," ``anti-terrorist," etc. Anarchism is a product to be sold on the market. So what if everyone is confused about its actual meaning, or the consequences of anarchist practice in society? Some of these ``anarchists" are more interested in, say, the ``Free Market" than they are in anarchism and are just playing both sides of the street.\\
The arrogance of this form of anarchist authoritarianism postulates an ignorant public, well manicured for these salesmen of anarchy by the social conditioning of the present culture. Why not take advantage of these facts and manipulate the public ourselves, just like everyone else does? Too bad for the manipulators, but people are smarter that these types give them credit for being.\\
A third type of anarchist authoritarian is more sophisticated than the other two. These types seldom participate in discussions in the SRAF bulletin, so a reference to that document may be unavailing. This type is really a synthesis of the other two, and an intellectual improvement. If the first two types of authoritarian anarchists were the ``immaculate conception" type and the ``sales hype" type, this third type might be called the ``embarrassed intellectual."\\
This type does not deny the value of internal education, but insists that discussion must be ``deep" or ``significant," and above all, ``correct." They are embarrassed by any evidence that anarchists are not perfect. (My god! People might even associate ME with these other people who call themselves anarchists!) They are embarrassed by the honest attempts of people (less capable than they) who are struggling with their anarchist perspectives and their efforts to articulate these perspectives. They express the opinion in many ways that if anarchists must describe their perspectives in public (even to an all-anarchist public) that their discussion should not contain any embarrassing errors. Such errors should be edited or censored from any publications so that no one can discover that maybe all anarchists have not developed their perspectives to a consistency equivalent to what Kropotkin achieved. I used to think Marxists were the only ones afraid to take a position on any question until they were sure they had the correct position. Anyway, insistence upon perfection before one acts or opens one's mouth certainly is not leadership, because the definition of perfection is socially determined as a result of the interchange among people and with the material conditions the face. These embarrassed intellectuals know, of course, what is and is not significant and worthy of putting forth to the public The inference is that we should defer to their judgment in theoretical matters.\\
Aside from whatever information appears in the SRAF bulletin, perhaps all anarchists cannot benefit from the discussions of our perspectives there. I refer to that bulletin in my article here because it is the only place I can think of where one can taste the expressions of all varieties of persons who are anarchists on this continent. Men and women (though mostly men), the very young, the very old, professional students and other academic intellectuals, semiliterate, craftspeople, blue and white collar workers, unemployed, professional or middle-income people, and welfare recipients. All these people treat each other as equals in dignity and respect and try to develop the experience to help each other build the means to create a free society.\\
I am convinced our own internal education is fully as important as whatever education benefit we may contribute to society as a whole. No matter how embarrassed one might be by the expression of, and attempts to articulate, our perspective by others, the process is very important for us all. Anarchists are not perfect. We have no infallible leaders to follow. We need to develop ourselves and our perspectives as much as anyone in this society We are as much affected by social conditioning as the most contented bourgeois. The difference is, we recognize a need for change and are struggling for the means.\\
Anyone who tries to prevent this process of education or to assert that it is unimportant, or worse, is really acting in an arrogant, authoritarian manner. Such an assertion is really an attempt to coerce those of us who are struggling with our perspectives to defer to the authority of the intellectuals, or some other ``standard-maker." I have benefited greatly from the mutual process which has been taking place in the SRAF bulletin, as well as from similar types of discussions in face-to-face meetings. Even though I recognize the bulletin is not agitational literature in itself appropriate to hand to non-anarchists on the street, the bulletin is also a tool and a source for ideas we can use to develop our own leadership potential so that we may provoke, influence, and entice others to develop their own initiative and self-reliance in society as a whole.\\
This is the kind of leadership we need to develop. If we do develop it, the ``leaders" will cease to exist. I hope by this process we can all become more fully aware of the authoritarian distortions produced in us by our conditioning. Any arrogance or self-righteousness about our own perspectives has got to be a product of this destructive conditioning. I hope we cease putting each other down for our inability to express ourselves in the best way possible. Our organizational efforts should be designed to provide us with solidarity and mutual support in our struggles --- both internal struggles with our education and perspectives inside our organizations, and external with the forces of reaction and confusion. Libertad o muerte.