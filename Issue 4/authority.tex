\chapter{Authority: Theirs, Ours or Nobody's?}
\chapterauthor{Canada Lynx}

\noindent Giovanni Baldelli, SOCIAL ANARCHISM (Penguin: Harmondsworth 1972), 192 pp.\\

\indent When this cat picked up Giovanni Baldelli's paperback on ``social" anarchism for only a buck, he was hoping it would answer George Orwell. For Orwell saw any no-government movement as merely substituting a closer and more efficient social tyranny for a more distant and less efficient political control. Or as Stephen Vizinczey puts it: ``Bigness is weakness... A world government, far from bringing about universal peace and stability, would mean total anarchy." Small groups can be the most tyrannical of all. Concerning Israel's communes (the kibbutzim) Amos Alon observed: ``There is little that can be as demanding and restricting to an individual's liberty as service to an extreme form of libertarianism."\\
For most of us living in North America political oppression is not an everyday reality. What \emph{is} real is our daily dose of social oppression: the ``old man" taking it out on his ``old lady," parents battering kids, teachers discriminating against their students, and the boss ripping off the poor slob ``lucky" enough to find work. All this shit is far more real to most of us than the Man with his cops, courts and taxes.\\
But if you're looking for relief from social oppression you'll find cool comfort in Baldelli. For Comrade Baldelli believes that authority is a Sugar Daddy and not a Dirty Old Man. Here we find our author in striking contrast to Errico Malatesta who simply speaks of anarchy as ``a society organized without authority."\\

\section*{Ethical Authority = Compulsion}
Baldelli dreams of an authority that in addition to its coercion and the power to give orders, will command ``respect." Perhaps it is an authority that know ``how to win friends and influence people (Dale Carnegie's book on human manipulation was a best seller in Nazi Germany). But William Godwin noted that authority stifles the development of our intelligence --- which is nothing less that our ability to choose. And Pierre Proudhon declared, ``Authority, Government, Power, State --- these all denote the same thing --- each man sees in it the means of oppressing and exploiting his fellows." Then Leo Tolstoy denounced ``the schemes of the possessors of authority --- nay, their unconscious effort --- is directed toward weakening the victims of their authority as much as possible; for the weaker the victim is, the more easily can he be held down." Now Baldelli informs us that in his ``anarchist" utopia ``an offender who refuses to accept and carry out a decision by an ethical authority, whether of `arbitration' or `assessment,' calls down upon himself the use of compulsion."\\
If this be anarchy, Comrade Baldelli, then anarchists can be only a very mixed-up clowder of crazy cats!\\
Still, I prefer to think that the only person who is really confused about every anarchist's response to authority is Comrade Baldelli himself. And more's the pity. But for this fatal flaw in his thinking, this could have been the book we've all been awaiting. For Baldelli says many good things and often says them well. Still the thoughts of Comrade Baldelli are only tinsel on the poison ivy vine when we consider that his misplaced reverence for authority is deadly to human freedom. Authority is the enemy of freedom and not its partner in joy and sorrow.

\section*{Ethical Capital: Baldelli's ``Karma"}
Like the Hindus with their notion of person's ``karma" --- a sort of divinely kept ledger book of one's good deeds --- Baldelli expands this thought to the whole of society under the name of ``ethical capital" --- a kind of collective ledger of everybody's good deeds that makes it possible for us to trust each other. So if too many of us drain society's bank account of its ``ethical capital" by committing too many rotten deeds that destroy trust, we either have a week government on the verge of collapse (what editorialists fancy as ``anarchy"), or a totalitarian state. In the latter event government then becomes organized mistrust where everyone in the ruling circles spies upon and knifes everyone else.\\
And yet for a fascist --- or even a Marxist --- tyranny to survive, there must be a minimum of ethical capital. Without a foundation of good, evil cannot flourish. Such is the notorious ``honor among thieves." Otherwise the leaders would all be assassinating one another (something that the CIA has well begun) and political power would vanish. That seemed to be happening with the U.S. before the departure of Tricky Dicky.

\section*{More Trust, Less Government}
Now if effective government organizes mistrust (by divide and rule) on a foundation of trust, then an anarchist society would have to organize itself on pure trust by uniting what Baldelli calls ``competence" with ``interest," that is the producer with the consumer. This might be done with friendly neighborhood affinity groups. Doing things with people you\\

\noindent ... \emph{[PAGE MISSING]}\footnote{If you have access to page 4 of the original publication, please contact us at anarchao--at--discordia--dot--fr.} ...\\

\noindent society --- including especially ``the right to decide"? Principless --- yes; reasonable expectations --- of course; predictable behavior --- sure; but ``rights" --- never! Who would enforce these rights? For laws to exist there must be lawgivers and law enforcers, and for rights to exist there must be authorities who ``permit" us to exercice them. Anarchists believe in a \emph{free} society, and not in a permissive society. For there is no freedom where there is the power to permit us anything. Those who would give us ``rights" can as quickly snatch them away as Mr. Trudeau did under the War Measures Act by suspending the Canadian Bill of Rights so kindly conferred upon us by John Diefenbaker.\\
Anarchists do not believe in written law because the written law is an instrument of tyranny --- never of freedom. We believe in the \emph{living} law --- the law of deeds, not words, the unwritten law of custom with its ``decent respect for the opinions of mankind." To the extent that we have to make agreements in writing, to that extent we indicate our distrust of one another. The more laws that are passed the more our ethical capital shrinks. And so we come to embrace government and trust in it because we have learned to distrust ourselves.

\section*{Baldelli's Technocrats}
In place of our more remote officials, Baldelli would give us local ``authorities" who exercise their powers on the basis of competence of ability (experts or technocrats) over those with a competency of interest (consumers). Yet in Baldelli's book competency of ability is superior to competency of interest. If I say to Baldelli's shoemaker, ``My shoes pinch," he could reply, ``Fuck off! Everyone says I make beautiful shoes. If you don't like them, make your own or go barefoot." Or as Henry Ford once put it about his Model T: ``Sure, you can have any color you want as long as it's black."\\
No. Technocracy, however decentralized, is no answer, Comrade Baldelli. It looks as if you had chased government out through the front door --- on the basis of incompetence --- only to bring it back through the rear door --- on the basis of competence.\\
True, the experts as producers would be less remote. We might even enjoy a dip in their swimming pool (though I'd check them first for sharks) sooner than we'd be invited in by Pierre Trudeau for a swim with Margaret.

\section*{Baldelli's Banditos}
Of course these technocrats (Mr. Baldelli ominously calls them ``guardians") would never have the power to give orders to kill anyone. That would be left to Baldelli's Banditos (he calls them ``an emergency corps") who would root out and, if necessary, kill the killers who threaten the anarchist (so-called) society. How the ``Emergency Corps" would function without orders to kill, Baldelli doesn't say. Maybe somebody would do the job whenever the spirit moved him. (A ``spontaneous" lynching in the grim manner of Friedrich Dürrenmatt's \emph{The Visit}?)\\
In Baldelliland we would see ``clandestine organizations" that ``should consider as no longer a member, but as an enemy, anyone who breaks down authority or abuses the meek." Well, Scarlet Pimpernel, move over. Here come Baldelli's Banditos. Of course, those so unfortunate as to incur the wrath of Baldelli's technocrats could be banished from Baldelliland, stripped of their worldly goods and possessions and not a little of their human dignity.\\
Sounds all too familiar, doesn't it? Not that Comrade Baldelli isn't a man of great personal charm and good will. He is all this and more --- a gentle schoolmaster trying to get some semblance of order out of a disorderly world and a scurvy lot of anarchists. And admittedly we rebels with a cause are an ornery, disorderly lot who are more like cats than dogs. We will not fit into anybody's tidy utopia without scratching the furniture, ripping up the cushions, peeing on the carpet and creating nasty stinks. And the rest of the time we'll yowl when confronted with prophylactic paradises: ``This is where we came in and we'll shit all over you unless you let us out!"

\section*{Anarchy Based on Relativist Ethics}
In ethics, then, Baldelli is a traditionalist and so does not recognize the decline and fall of the absolute. The twentieth-century revolution in axiology --- the triumph of relativist ethics --- has passed over Baldelli as a nineteenth-century holdover. Although anarchy demands the utmost of all of us in the acceptance. and appreciation of a whole rainbow of lifestyles, Baldelli is still stuck with the obsolete Golden Rule with its ``do-as-you-would-be-done-by" absurdities. Most people with anarchist tendencies would more likely agree with George Bernard Shaw who once wittily suggested: ``Don't do to others as you would be done by. Their tastes may be different."\\
I thrive on meat, but a fellow rebel is unsure of any cause except his ``no-egg-near-my-face" vegetarianism. Since I seldom see him I don't mind in the least preparing him a vegetarian dish. And in the unlikely event that we should live in the same household we'd probably agree on a smorgasbord.\\
A trivial example? Hardly. A pluralist anarchism means a lively appreciation for and interest in the widest possible variety of non-ripoff lifestyles.

\section*{The End-Means Problem is Phoney}
Baldelli seems fascinated by the Kantian dogma that ``the end does not justify the means." This idea must have some religious meaning for him, for he doesn't worry too much about violence (as means) in self defense (as the end). The relativist considers the ``end-means" problem as a phoney issue. Translate it into the question: ``Is the object (as end) never worth its price (as means)?" and then at once you'll see how silly it really is! As an ethical rule ``the end never justifies the means" would paralyze all exchanges of goods, ideas, or anything else that might require any effort.\\
A friend who read this felt that I was being a little unfair to ``end-means" enthusiasts. What they really meant was that the means must be \emph{consistent} with the end. But so what? Since authoritarians have a stake in a violent society with all authority resting on mental and physical violence, violence is a perfectly consistent means to a violent end! However, there is something more sinister behind all this.\\
Many of us with anarchistic tendencies deplore the violent society and its creature, the state. If we were to adopt the rule that our means must be consistent with our end, 
the non-violent society, then, of course, we must refuse any violent methods to bring about the loving society. This self-denying ordinance means that those who oppose authority must surrender their option for self defense and so submit to the aggressor's pleasure. Such is the ``slave-morality" of the Christian Church and many Eastern religions that Nietzsche long ago exposed but which fascist pacifism imposes on some misguided people.

\section*{Fascist Pacifism Drains Our Ethical Capital}
         Pacifists who are wittingly or unwittingly prodded on by the authorities to resist not evil obviously draw heavily on the ethical capital built up by courageous men and women who chose not to surrender but to resist oppression by the most effective means available to them. If fewer people resist tyranny anywhere it will endure that much longer. It does not matter whether the tyranny appears on a mass or political level or on a small group or personal level. In the latter case I would not hesitate to stop a burglar, rapist, or murderer from entering my house and threatening those I love if the means of violence were readily at hand. Yet i would still be at peace with my neighbors. The end-means ethic that Baldelli endorses would forbid my doing this if words mean anything at all. And of course if I failed to use whatever necessary force was at my command the ethical capital among those who trusted me or would trust me in their defense would be diminished. Burglars, murderers, and rapists would be 
assured of easy pickings next time.\\
Such is the reactionary nature of the end-means problem that it would disarm the defender while arming authority as the aggressor. Baldelli's acceptance of the pseudo-ethics of the end-means purveyors would deny all victims of the option of revolution against the oppressor --- an option that not even conservative statists like Thomas Hobbes and Edmund Burke and liberal statists like John Locke and Thomas Jefferson would deny to a populace with unredressed grievances.

\section*{No Right or Wrong --- Only Consequences}
All of which leads many of us to the conclusion that when dealing with authority, ``There is no right or wrong; only consequences." As Alex Comfort has observed, ``Authority attracts delinquents." It attracts the worst and most deeply corrupt element in ourselves. The exercise of authority --- of domination over others --- is sociopathic. The incorrigible is more likely to be the judge than the person in the dock. Only the bad are attracted by authority. The good will always reject power over the weak, but they will not hesitate to use any power they can muster against oppression wherever and whenever they can find it.\\
The nature of this oppression as it affects the United States has never been better described than by Robert Townsend, a former corporation executive with Westinghouse:
\blockquote{\textsc{America is run largely by and for about 5,000 people who are actively supported by 50,000 beavers eager to take their places. I arrive at the round figure this way: maybe 2,500 megacorporation executives, 500 politicians, lobbyists and congressional committee chairmen, 500 investment bankers, 500 commercial bankers, 500 partners in major accounting firms, 500 labor brokers. If you don't like my figures, make up your own. We won't be far apart in the context of a country with 210 million people. The 5,000 appoint their own successors and are responsible to nobody. They treat this nation as an exclusive whorehouse designed for their comfort and kicks. The president of these United States, in their private view, is head towel boy.}}
(Knock off a zero and change a few titles and this description could just as well apply to Canada --- Lynks)

\section*{Authoritarian Ethics Helps Oppression}
Traditional ethics, which is, authoritarian ethics, has always been on the side of the rulers against the ruled. Absolutist ethics is, of course hostile to anarchist ethics which is relativistic and based on human experience rather than upon the decrees of a divine or human authority. Free people do not believe anything to be so because someone says so. Our only authority is that of our experience coupled with the experiences of those whom we have reason to trust. Even so, we can reject the experiences of others and chance the consequences. Freedom must include the option to make ``mistakes," or it is not freedom.\\
Smith Senior (as an authoritarian parent) says to his son, ``When I was your age I made the same mistake." Smith Junior (the rebel) replies, ``I'll know it is a mistake when I am your age."

\section*{Can Anarchists Be Ethical?}
``True" anarchists can discuss what is right and wrong among themselves because they have a highly developed ethical capital, a bond of trust and a sense of solidarity that makes the concepts of ``right" and ``wrong" meaningful.\\
This is not to say that your ``true anarchist" is an elitist, a person who will rank himself among the privileged 5,000 in the United States or top 500 in Canada. Quite the contrary! The ``true anarchist" overflows with love, is more compassionate, more creative and therefore exquisitely sensitive to the ``still sad music of humanity." But where are such exotic creatures as ``true anarchists" to be found?

\section*{Strugglers and Stragglers}
``I can't relate to what you've written... Your ideas are way up there and we are way down here," exclaimed two acquaintances of mine who were struggling through ``busywork" to create what they fancied to be an ``anarchistic" collective. Yet these were the very people who had borrowed expensive tools from a comrade and lost them. As average freaks who had wound up on welfare (what Marx derisively referred to as the lumpenproletariat) their dope dealings and their petty rip-offs were no better, perhaps, and no worse than the rest of us.\\
Discussions on anarchist ethics or the behavior of ``the true anarchist" could not meet their need. Language had so long been misused in their lives that words had lost any meaning for then. ``Come live with us," they seemed to be saying, really, ``we'll treat you like shit. Milk you for everything you've got and take it as our due, for anyone better off than us is our oppressor. And once you are no better off than we are --- and can still take it --- maybe, just maybe, we'll look at your words."\\
But I knew then that I was no more a ``true anarchist" than they were. Today the existence of the ``anarchist man" and the ``anarchist woman" is an impossibility. And to speak of ``anarchist ethics" among all of us who possess perhaps a few anarchistic tendencies when someone is not being conned is as pointless as a medieval treatise on ethics among angels.

\section*{The Divided Self: The Origin of Our Oppression}
Yet if words have lost their meaning for even those among us who are still groping for some kind of enlightenment, then is the quest through words --- through publications such as \emph{No Governor} even --- a fatuous enterprise? I prefer not to believe so despite the terrible irrationalities, stupidities and cruelties of our sadomasochist society. Perhaps we can discover a few thoughts buried in the rubbish heap of our culture that even those with slight anarchistic tendencies can apply to their lives and ultimately effect the transformation of humankind.\\
How then might an anarchist ethic emerge from our minimal contractual ethic --- an ethic overflowing with spontaneity and love, an ethic that rests on complete trust and complete faith in one's fellows and makes no conditions?\\
Perhaps we'll have to backtrack a little in our thinking and, putting Baldelli aside, see how the ``true anarchist" might emerge, for ``the beautiful children are not yet born."\\
One thing I've learned from my own life as well as from my observations of the lives of others is that the most terrible alienation of all is the separation of the mind from the body. Those of us with anarchistic tendencies may be quick enough to denounce authority whether it is exercised by parents, teachers, police, the priest or the boss. The authoritarian pig in others is easily recognized. But it is less easy for us to recognize the pig in ourselves when we push others around with words or blows, and still harder when we let our minds dictate to our bodies. As lily West wisely remarked, ``We can not hope to make a revolution without overthrowing the government in our heads."

\section*{Our Mind-Shackled Bodies}
Authority really gains the upper hand when our parents and society split apart our minds and bodies and then made our ``presumptuous brain" dictate to our body selves. Deeds did this to us from our infancy; words confirmed it from our childhood. Many of us were bottle-fed rather than breast-fed --- but that was only the beginning. Instead of the warmth and long periods of cuddling we might have gotten from both our parents' bodies we soon enough learned to be ashamed of our most pleasurable sensations and gradually learned to push all our erotic feelings below the belt. As the mind took over, our bodies began to become rigid and even lose full sensation in various spots, ``Straighten those shoulders! Pull in that gut! I'll give you a penny if your buttocks will hold it. Having trouble again? Time for an enema!"\\
Gradually our body-selves lost their coordination, lost their grace, and even on occasion lost their beauty as we took out the mind's misery on the body so that we got fat or burst out in pimples. So if we with anarchistic leanings haven't overthrown the government in our heads, are we not the world's worst hypocrites? Perhaps so, but in a society dedicated to total hypocrisy, both conscious and unconscious, we will be the most aware hypocrites around if we persist in asking ourselves the right questions.

\section*{Big, Fat Orgasms for Restored Bodyminds}
Many years ago Carl Jung observed: ``If we are still caught by the old idea of the antithesis between mind and matter, the present state of affairs means an unbearable contradiction; it may even divide us against ourselves..." Unfortunately, though Jung had stumbled upon the greatest discovery of his life, he simply picked himself up, dusted himself off and went on for the rest of his life as if nothing had happened. But another of Sigmund Freud's students fared better. Wilhelm Reich was to carry the discovery further by reuniting mind and matter in \emph{The Function of the Orgasm}.\\
Just as Karl Marx believed that human energy was primarily rooted in economics --- in the workplace --- so Reich held that our energy was rooted in the genitals. Naturally Marx got a more respectful hearing. But where Marx would have surplus energy in the form of surplus value returned to the producer, Reich found surplus energy generated by complete or partial orgasmic release to leave the happy lovers overflowing with vigor and feeling renewed. Having temporarily released the internal repression in their bodies, the lovers had been set free. The more one made Reichian (but not neurotic) love, the more energy a person had to make creative revolution. Thus ``sex economy" was born to so outrage the emotionally ill authoritarians of a generation ago that Reich died in a U.S. prison.

\section*{Sexual Liberation and Anarchy}
Anarchistic writers have never been too comfortable with discussions on sex. Outside of Albert Libertad, Emma Goldman, Paul Goodman, Alex Comfort and Abbie Hoffman, the ``classical" anarchists have given the subject scant treatment. Baldelli writes a whole volume on social anarchism but he dismisses sex in only a sentence or two.\\
Yet the most serious and scurrilous charge that was ever made against the integrity of the anarchistic revolutionaries in Spain was their alleged murdering of ``male prostitutes" in Barcelona. This later turned out to be an atrocity tale concocted by the supposed anarchist scholar, George Woodcock. Doubtless this renegade hoped to get himself thereby into the good grace of the establishment. (No wonder Woodcock's guilty conscience keeps him well away from Vancouver's would-be anarchists. He could be entirely discredited by them and his career as an ``anarchologist" spectacularly ended.) Like nudists who have made their one bold gesture against convention and become reactionary about everything else (including sex), so people who call themselves anarchists are often especially careful not to make any waves in sexual 
 waters.\\
Now the champions of authority have never been so backward about linking anarchy with sexual liberation! By so doing, authoritarians of course wish to discredit \emph{both} sexual liberation and anarchy. Our authoritarian critics equate love among the free with ``sexploitation." But the only form of freedom that authority can imagine is license, and license obviously spells human exploitation. Over thirty years ago an outraged enemy of Wilhelm Reich christened his diatribe \emph{The Cult of Sex and Anarchy} (\emph{Harper's Magazine}, March, 1945).

\section*{``Beyond Sexual Freedom --- Anarchy!" --- Socarides}
Today our authoritarian critics are still at it. The latest big gun in the sexual backlash of the seventies is Manhattan psychiatrist Charles W. Socarides. ``Beyond sexual freedom lies madness and anarchy!" screams this incredible shrink. But if ``the communards of sexual liberation are antifascist, pro-Maoist China, pro-black power, antijunta-dominated Greece and even opposed to the Vietnam war," to Socarides' horror, then perhaps sexual liberation deserve more attention from the would-be total liberationists of anarchy.\\
Although no anarchist himself, Wilhelm Reich made a major contribution to our enlightenment on sex by focusing on body language as a clue to the uptight authoritarian personality. It didn't take long for Reich to see that a rigid mind commands a rigid body (or is it really the other way around?). Loosen the body by massage, restore sensations in the pelvis and in areas long anesthetized by the commands of external authority, and one would begin to loosen the hold over natural feeling now exercised by our presumptuous brains. Very soon we'd not only feel better and healthier, we'd even get big beautiful heterosexual orgasms rather than itty bitty masturbatory or homosexual climaxes. Where the orgasm would leave one deeply satisfied and full of energy for the next day, the minor climaxes of tense, neurotic sex left partners irritated and dissatisfied, mentally and physically exhausted and bored with each other. Needless to say what mostly passes for sex, especially among sophisticates in our society, ranges from the total failure of male impotence and female frigidity through petty masturbatory climaxes and partial clitoral and penile orgasms. So aim for the total bodily orgasm, said Reich. Great stuff, if it happens!

\section*{The Elusive ``Big O"}
But pursue it and the ``Big O" disappears! By zeroing in on the total orgasm as the goal rather than as a byproduct of bodily and sexual health, Reich was paying too much attention to orgasm as the capitalistic payoff of ``the performance principle." Reich was right about beautiful bodily orgasms as being signs of sexual health. But the pursuit of the ``Big O" becomes its commercial namesake. Like happiness, the ``Big 0" of bodymind integrity, cannot be bought.\\
Why? For here again in trying for Reich's ``Big O"  we do not have two bodyminds melting together into a sunburst of ecstasy but rather two separate minds dictating to their respective bodies and to the bodies of their partners. So long as the authoritarian split between mind and body prevails, the individual can only strive for but seldom achieve the longed-for ``Big O". No amount of reading and practicing Alex Comfort's \emph{The Joy of Sex} or working on \emph{More Joy}, no amount of swinging or of homosexual or bisexual activity, no amount of sexual sophistication and experience will unite our minds with our bodies even as we try to unite with the bodies of others.

\section*{The Fear of Incest Split Our Bodyminds}
Freud long ago recognized that the Oedipus complex that originates in the incest taboo is the main binding and repressive force in the human personality. As children  first learn how to love and to hate in the family, so they learn, too, how to both love and hate the same persons at the same time. First it's with our mothers and fathers and then with our sisters and brothers. The hate aspect of our relationship is created by the incest taboo. But the prohibition against touching family genitals is to Freud the price we must pay for civilization. Actually, Freud equated civilization with sadomasochist or authoritarian culture, which is quite a different matter. So Freud was allowed to talk freely and write about sex (although he could do little to relieve sexual anxiety having thus rendered his hostages to authority).\\
However, it remained for the psychologically trained anarchistic Yippie, Abbie Hoffman, to go one step beyond both Freud and Reich. ``When the Oedipus complex bites the dust," Hoffman declared, ``then our children and grandchildren will be practicing polymorphous (all combinations) incest." How this was to be achieved, Abbie didn't elaborate before ducking underground. But one suspects that Abbie's remarks were more a part of the good old Yippie tactic of freaking out the media than the result of any serious conscious analysis. Haphazard incestuous relationships more often create guilt, build up unbearable tensions and climax in suicide and homicide rather than make for well-rounded, happy, healthy human beings. Still, Abbie brought back the concept of sexual liberation into the anarchistic court where it had too long lain dormant. It remained for that archangel of anarchy, Canada Links, to forge the final link to the solution that Freud evaded, Jung recognized, Reich responded to and Abbie Hoffman playfully fumbled.

\section*{The Body Doesn't Lie}
Alexander Lowen, M.D. (a student of Reich) made a most important observation for would-be anarchists when he saw that while our minds will often deceive us as well as others, our bodies resolutely refuse to tell lies. He called this phenomenon the truth of the body.\\
The more astute among us unconsciously check the truth of our bodies in a handshake, for we know that the words of our minds are often the vehicles of deceit, whereas body contact tells us what we really feel about each other. Our bodies most commonly betray us when we lie through the increased sweating of our palms and sudden variations in our breathing and pulse rates, facts that the older lie detecting machines could graphically make visible. The stress in our voices betrays our lies, telling a computer (and many an ``intuitive" woman) that we feel tense about what we're saying.\\
It therefore follows as day follows night that if our bodies are really together with our minds, our minds can never decieve ourselves or anyone else. If anarchy means total liberation and truthfulness, and if total liberation rests upon total trust, then to be liberated means that we must possess entirely truthful and complete bodyminds. So to develop the anarchist society based on trust we must first aim at becoming bodyminds again or at least preventing the bodyminds of our children from splitting.\\
When I was once holding a rap session on anarchy, a girl moved right next to me and gazed at me so intently that I became quite nervous. Finally she remarked, ``You're talking about someone else's thoughts and feelings. You don't share them --- they're not your own."\\
I was taken aback. ``How do you know that?" I challenged her. Before I could stop her she quickly put her right arm around my waist and then exclaimed, ``Because your body is tense all around here!"\\
I was shocked by the experience, but my lying mind came to my rescue with a ready-made rationalization: "She's just another hippie anti-intellectual on a mystical trip." But of course she was dead right, while I was simply dead.

\section*{The Case of the Impotent Liar}
A more common experience of the truth of the body in the face of a deceitful mind is that of a man whose erection suddenly fails him just at the point of intercourse. He may protest, ``But I love you, honey. It's just that I've had a trying day. Guess a hard day's work got me down."\\
``We'll see," answers the knowing you lady and she quickly sucks him back into an instant erection. ``Very interesting!" she observes. ``If you had been really tired rather than turned off by me, you'd have no trouble keeping that hard-on. A change would have been as good as a rest."\\
Women obviously have an advantage over men here, as the truth of men's bodies is embarrassingly visible in sexual failure. Many a woman has had occasion to exclaim, ``If I had to have an erection every time I made love with my husband, he'd have murdered me years ago."\\
True anarchists, of course, never have to lie to each other. They would find lying quite impossible with their united bodyminds. Only in a society of people with unified bodyminds is an anarchist ethics --- an ethic based on absolute trust rather than distrust --- really possible.

\section*{Total Trust Necessary for the Anarchist Ethic}
For all of us who have had our minds separated from our bodies from early infancy (and who hasn't?), perfect trust as the prerequisite for any true anarchist ethic is really impossible. To merge our minds and bodies together again so that it can be truly said that our bodies think would be difficult at best. Most likely, for those of our generation whose bodies and minds have been separated, our attempts to regain bodymind congruence at all times would end in failure. (Even more alienating are the ``spiritual" trips in which the mind is further separated into ``spirit" making it even harder for people to get together with the now consciously despised body! So working with adults whose bodies and minds have become alienated, and whose minds may have become even further alienated from the body into spirit, is much like teaching a polio victim how to walk again or a basket case how to use artificial limbs. Still, if our minds and bodies will never permanently merge again, as they did in early infancy, we can doubtless bring back a greater congruence between them even if we can only emotionally and mentally hobble about. But surely this is better than our present paralysis and almost complete dependence on the cop inside our heads!

\section*{``The Beautiful Children are Not Yet Born"}
If we live with e mate we can talk over our problems as honestly as we ore able and learn to massage each other regularly and generally and then erotically while we talk. If we are alone we'll learn to massage ourselves both generally and erotically while thinking good thoughts about ourselves. Even then our bodies and minds will never attain perfect congruence... except perhaps for brief moments when we completely ``lose ourselves" in the moment of partial orgasm. The achievement of the permanent bodymind dynamic may only happen with our children or our grandchildren.\\
Those of us who may be discerning and lucky enough to find someone with whom we can have and raise our children very differently will find jointly-shared child raising not to be a chore but the most liberating experience in our lives. Such people would enthusiastically agree to massage their kids from infancy just before the children go to sleep. The parents together would talk over the kids' dreams with them as they massaged their bodies, suggesting more positive dreams Involving love rather than fear. One would expect the kids to grow up with beautifully integrated bodyminds capable of far more love and for less hatred than people of our generation have for each other. These kids would be far more sensual, truthful, open, free, graceful, beautiful and precocious in every way than their parents.\\
Unfortunately, most of the-people I have met professing anarchistic tendencies are really authoritarian individualists when it comes to how they treat or regard children. Some have, fortunately, decided not to have kids at all. Children born of these professional ``anarchists" would probably be ``battered babies" rather than ``stroked babies." And ``battered babies" are likely to become ``battering parents" exhibiting the worst forms  of authoritarianism just as ``stroked babies" are likely to become ``stroking parents" exhibiting the best forms of libertarianism. If children are a burden to your freedom, then, of course, it's much better not to have them. Yet not to 'want children is not so much an expression of ``freedom" as it is an expression of the degree of the alienation of our minds from our bodies.

\section*{Can People of Anarchistic Tendency Be Trusted Any More than Anyone Else?}
Undoubtedly some will sincerely object that erotically-fed families would become terribly ``incestuous" or ``ingrown" --- the kids becoming entirely dependent on their parents. Those who believe that independence comes when a mother bear cuffs her cubs are, of course, absolutely right --- for bear cubs! But when this cat read of a humanist family down in Texas advertising for partners to join their family, he knew there was already an intelligent human rather than a moronic animal solution, to this problem.\\
Moreover, kids with integrated bodyminds are likely to have incorporated a far greater degree of their unconscious ids into their conscious egos with a resulting higher level of awareness. Far from restricting freedom and independence one would expect that those with integrated bodyminds would mature much more rapidly and develop far beyond their age peers raised in an authoritarian sadomasochist culture. Undoubtedly childhood ``immaturity" is artificially prolonged by the wasted psychic energy involved in repressing the Oedipus conflict during the ``latency period."\\
Today we with anarchistic tendencies may or may not be more trustworthy than anyone else even if we exclude the authoritarian individualists who look upon their own freedom as a license to rip off others. A desire for freedom that argues only for unreliability rather than spontaneity in social relations weakens trust and makes nonsense out of any commitment. Surely if one's commitments are the sort that bind one or are at least felt to be too binding, one shouldn't make them in the first place. But I've met too many people who call themselves anarchists who show a greater than average inability to keep their promises even if it is a simple matter of showing up or producing work on a project on which that person professes some interest or enthusiasm.\\
Of course there are the notable exceptions. These people are the salt of the earth, whom one can also find in authoritarian organizations. But the word anarchy inevitably 
attracts not only the occasional unstable fanatic but the paid provocateur and informer to make sure that dissidence will never become effectively organized. One of the few ``triumphs" of the technological society has been social engineering, and because anarchists in the past have always possessed the most creative ideas for social change they have attracted more than their share of attention from the authorities. So in practice this means we can only apply a minimal, ethics with our ``comrades" as we would with strangers and enemies --- a contractual ethics of consequence which will have to remain operative in most cases until we develop a new generation of people whose bodies have not been separated from their minds.

\section*{Human Potential Trapped by Incest Fear}
The transformation of humanity --- the great leap forward --- will likely come when enough of us are willing to lift the binding weight of the incest taboo that rests like a mighty Alp upon the human psyche. Until this superstition that undergirds all authority is critically recognized to be the blinding error that it is, all talk of any quantum leap in human potential (to say nothing of the realization of an anarchist society!) is idle chatter. Only people having integrated bodyminds can accomplish the anarchist transformation that transcends mere political revolution. For authority is bred into us with the birth of every babe.\\
Rebels there will always be. But most are miniature or would-be Jenghiz Khans, not the rare Emma Goldman or Peter Kropotkin. Baldelli is certainly right to say that ``rebellion without enlightenment" means no revolution," no change in the status quo.

\section*{The Dream People of Malaysia}
Given the overwhelming power of authority in our lives, is it possible for us even to imagine a successful, non-violent, peacable and happy society that all but power-seeking criminals desire to see for others as well as for themselves? The Senoi tribe of Malaysia may have already shown us a way out of the tunnel of despair. They have no recorded history of crime or mental illness, no violence, no police or courts and no jails.\\
How do the Senoi avoid the reasons for being aggressive and having an internal government? ``They talk about their dreams: At breakfast, lunch and dinner --- all the time," says clinical psychologist Dr. Harry Fiss of the University of Connecticut.\\
By learning to control their dreaming so that they have positive dreams, the Senoi help their unconscious minds direct their conscious behavior into constructive channels that result in a very loving, peacable people. So, Comrade Baldelli: Do not despair! Anarchy may be closer than you think, even if it cannot be in the directions you suggest.
