\chapter{Doing Anarchism Yourself}
\chapterauthor{Robert Shea}

Deciding to call oneself an anarchist is usually the end of a long process of eliminating other philosophies and the beginning of another long process of learning what anarchism really is all about. One new thing that has to be learned and is rarely fully appreciated is the role of the individual in anarchism. It's easier to give lip service to an abstract ideal of individualism than it is to act as if you actually believe that no one and no group and no principle is more important than you, yourself. Or more important than any other particular person. By this belief you commit yourself to no longer passing the buck to others. You declare your preference for a way of life in which you make all decisions and take all responsibility and expect to rely on your own thinking and your own efforts. This is a difficult ideal.\\
The anarchist movement in this country is very small and has difficulty making itself heard. This is due, not so much to any inherent implausibility in anarchist ideas, but rather to the problem that once people accept the ideas there seems to be no clear-cut direction to take. Groups come together and drift apart; publications appear and vanish; actions and demonstrations are staged and forgotten. Meanwhile, governments and state-supported economic systems seem to grow stronger. Sometimes it seems hopeless.\\
Here and there much is accomplished, often by one person or three or six. One group in one place will get its energies together and be printing and distributing literature, organizing, picketing, demonstrating, finding imaginative ways of getting the message across, setting an example for the rest of the movement. But in many places another picture is typical: Nobody knows what to do. Meetings are boring and seem purposeless. If new people come into the group they soon drop out for lack of promising projects to work on. Constructive suggestions or good ideas die on the vine because objections are raised or because no one follows through. It seems impossible to raise money. The group is divided by personality clashes. Publications die after an issue or two. The temptation is great to chuck it all and go to work for some local reform candidate or join some local Marxist group, just to have a feeling of accomplishing something.\\
Of course, the desire to accomplish something is itself a product of authoritarian thinking. Authoritarians are always measuring themselves by their numbers and the size of their accomplishments. During the first couple of million years of our species' existence on this planet we didn't accomplish much. Life for hunting and gathering people would seem relatively monotonous to a modern person. The really heavy demands for accomplishment must have originated with the domestication of man, with the invention of the state, when work and production quotas were set by priests and kinds and taxes were imposed. The hunting and gathering life would probably have been much more congenial to anarchists. The accomplishments of individuals acting on their own or in small groups need to be measured by different standards than those of disciplined mass organizations.\\
At the same time, it would be good if we could achieve more than we are doing now and if we could feel better about what we do achieve. Not for the sake of assuage guilt, not to feel that we're as good as various political movements, but simply to keep anarchism itself alive. Our main task today, as I see it, is not saving the world, but the humbler goal of keeping anarchism going, so that those who think as we do will be more numerous and more influential in the next generation than they are in this. If the race can manage to survive without our help for a generation or two more, there may come a time when we actually will have the strength to halt the doomward march of civilization and turn it around.\\
The sense of futility that afflicts some anarchists is due, I think, to each person's expecting others to come up with ideas, to waiting for someone else to make the first move, to seeking group approval before launching a project, to expecting others to take some of the responsibility and help with some of the work, to limiting one's own role in the group to the performance of some specialized function. In some people, this excessive reliance on others extends even to an unwillingness to define for oneself the aims and strategy of the movement. The result of this dependency is that nothing gets done unless some obvious project of fortuitous opportunity for action galvanizes everybody at once. Otherwise people blame one another for the group's failure to act constructively and eventually drift apart. Some turn to authoritarian movements of the left and right because that's where the action seems to be; others go into more conventional politics; still others subside into quietism. Or leaders may arise in the group and take charge. This can help for a while, especially if the leader is tactful and the group understands the practical need for temporary leadership. But in an anarchist group sooner or later a faction is likely to arise that will reject the leader just because the mere exercise of leadership, no matter how informal and non-coercive, is thought to be inimical to anarchism. In any event the emergence of a leader can be disruptive and, while postponing disintegration for a time, will lead to it in the long run. And the breakup of the group, unfortunately, often means the abandonment of anarchism by the members of the group as a philosophy that doesn't work. I think the real reason for these developments is not that anarchism doesn't work, but that it often isn't fully thought through and applied. What's needed are distinctly anarchist ways of getting things done.\\
There is an authoritarian pattern of group action that is pretty much universal regardless of the group's purpose or political-economic philosophy. Primary is the assumption that the group needs a leader and that very few people are capable of leadership. The responsibility for setting aims, coming up with ideas for action and assign tasks then belongs to the leader. Followers do not expect themselves to work as hard as the leaders, or to have ideas or display initiative. Among authoritarians the division of labor produces the impression that most work is unpleasant and undesirable, resulting in a demand that the drudgery be distributed equally among members of the drudging class. Everyone is expected to contribute an equal amount of labor to the group's well-being; there are to be no slackers. One of the tasks of the leader is to keep everybody busy, get everybody to obey orders. At some point in humanity's past it became customary to use force to control people. This was the origin of the state. As Jacques Ellul puts it in \emph{The Political Illusion}, `` To say that the state should not employ force is simply to say that there should be no state."\\
In authoritarian thinking there is an invisible line which people cross when they become members of the group. They have in their own minds, and in the minds of other members, surrendered a certain amount of personal sovereignty. The welfare of the group becomes more important than individual self-interest. It is this sacrifice of autonomy that supposedly gives leaders the right to discipline and punish.\\
Democratic organizations don't differ from this pattern, and such groups are also authoritarian. Although decisions are made and leaders appointed by vote, the leader-follower structure is the same, as is the claim of the supremacy of the group's will over the individual will. Once a decision is made by vote all members are expected to support it whether or not they agree with it. As has often been said, democracy does not do away with tyrants, it merely makes the majority a tyrant.\\
What happens when people reject authoritarianism? What they are rejecting, essentially, is the use of force to coerce the unwilling. This is the essence of government. In order to justify its use of force, government claims that the welfare of society has a higher moral claim than the individual will. To reject that claim means that you rate individual autonomy as the higher value and hold that a group or its leaders are never justified in forcing a recalcitrant member into line. (Of course, it is possible to take the position that the group's welfare is more important than the individual's welfare \emph{but} that the group has no right to coerce the individual. But I can't see on what basis the group could be so limited.)\\
Anti-authoritarian groups generally have no constitutions, no leaders, no formal procedures for reaching decisions. They are usually open to all comers and neither officially induct nor expel members. So membership and its obligations tend to be self-defined, or at least a matter for debate. The clear line which distinguishes authoritarian group members from non-members is missing. In many cases, however, these groups, though they have rid themselves of the trappings of authoritarianism, have retained authoritarian habits of thinking. Also, they have jettisoned authoritarian methods of getting things done, but they have not developed anarchist ways of doing things. For example, the idea that there are few qualified leaders and many followers often induces people to wait unconsciously for a leader, to expect strength and guidance from an external source, even though they had explicitly rejected the principle of leadership. Then there are those who spend all their time denouncing anyone else's initiative as an attempt to lead. Still other people are inhibited from showing initiative, fearing that they harbor within themselves impulses toward leadership.\\
Anarchists also carry authoritarian notions of the division of labor into their groups. This mainly takes the form of those who are productive resenting the non-contribution of less active members. One person may be working 16 hours a day while ten people who claim to be part of the group appear to do nothing. Since discipline is considered un-anarchistic, the usual way of coping with this imbalance is to harangue the less productive members, warning of dire consequences if everybody doesn't get behind the program. Haranguing, unlike coercion, is consistent with anarchist principles. Sometimes, if the energetic members get fed up with the invincible apathy of the others they may quit and let the group founder. Rarely do the hard workers expect gratitude, but the unfair criticism they are sometimes subjected to --- charges of trying to be leaders, gurus or superstars --- can sometimes be the last straw. Too often, instead of manifesting Kropotkin's ideal of mutual aid, anarchist groups destroy themselves through mutual recrimination.\\
Another hangover of authoritarian thinking in anarchist groups is the tendency to treat personality clashes and factional fights as catastrophic. Intramural conflict is as common as colds in winter in all groups, authoritarian and anarchist, in this culture. Among authoritarians such clashes frequently lead to splintering or to the annihilation or expulsion of one faction. But most authoritarians treat serious conflict as being intolerable. It is --- if one wants to achieve the machinelike unity and discipline that is the authoritarian ideal. But conflict should not be intolerable for anarchists. In fact, people are attracted to anarchism because they feel they should be able to fight in peace.\\
Anarchists also frequently retain the authoritarian idea that there must be formal approval for a project. Anyone with an idea in mind feels a need to get a consensus before acting. Frequently since majority rule is not accepted as a principle, nothing will be done unless there is unanimous approval. I have seen one person's objection stymie action on a perfectly sound idea because the rest didn't want to go ahead without total approval.\\
Right now, it seems to me, there is a flourishing mystique of excessive reverence for groups, of deification of the collective will. Team spirit is expected to replace leadership. In part this is the result of a feeling that the movement of the 60s was conned and ripped off by some of its leaders and spokesmen. In part it comes from a general revulsion from patriarchal authority spurred by the women's liberation movement. In part it seems to be an identification of the evils of capitalism with individualism. This belief in the superiority of the communal mind goes a long way back; in post-Revolutionary Russia there were experiments with conductorless orchestras (but not, so far as I know, with captainless ships). This attitude is understandable and healthy, but it assumes that there are only two ways of getting things done, authority or consensus. Carried to excess, this supposed respect for the group's opinion can mask laziness, apathy, timidity, inhibition, obstructiveness, unwillingness to take responsibility. Everyone sits paralyzed, waiting not for a leader but for a mythical entity to take charge, the group mind. And because there's no such thing as a group mind, the energizing impulse never comes. This worship of the communal spirit is authoritarianism without authority, all followers and no leaders.\\
It is simply very difficult to take the individual seriously. Over 10,000 years of authoritarian programming work against it. The I.W.W. anthem refers disparagingly to ``the feeble strength of one." Individual desires are small, mean, petty, selfish --- ``mere." Almost all moral systems are based on the belief that to be binding a moral code must be authoritative for everyone. An ethical theory based on the idea that each person's morals are binding only for him or her is dismissed as being bases on sentiment, whim, subjectivism. Much traditional U.S. rhetoric, like much anarchist ideology, pays lip service to individualism, but authoritarianism is rampant in this country. Ayn Rand's Objectivism, supposedly the individualist ideology \emph{par excellence}, rigorously throws out heretics and dissenters. In the mysticism that has captured the imaginations of many bright people, ego is a dirty word.\\
And yet, individuals are really all the anarchist movement had to work with. A big step toward learning to be an anarchist is to look for and follow the light withing. Only when anarchists start taking their own individuality seriously will the movement get off the ground.\\
If a meeting seems boring and purposeless, why blame the group and wait for someone else to do something? Require yourself to produce positive proposals and speak out. Don't wait to be told what to do; think of things for yourself to do and announce that this is what you are going to do. If you need help, ask for it. If you get help, you have group united by a common project, rather than a collection of people bound only by an amorphous commitment to anarchism and a vague agreement about the terrible state of the world. If no one wants to help, then you, the person who had the idea, should try if possible to carry it out alone. Anarchists have to learn self-discipline and self-sufficiency.\\
Don't expect the anarchist movement to match the performance of authoritarian political organizations. From the point of view of real freedom, the marvels of capitalism or communism mean little. The source of real creativity is the unfettered individual mind, and its accomplishments permeate a culture subtly, without the help of marching masses, disciplined parties, overwhelming technology or armed force. Let us be content to accomplish things on a small, local, human scale --- a Cro-Magnon scale --- believing that such achievements take deeper root and have more value in the long run.\\
Those who have learned to be thoroughgoing anarchists do their work because they enjoy it. They get pleasure from the excitement, the commitment, the hope for the future. They are full of ideas of their own and not terribly concerned about what others think of them. If they think something is worth doing, they'll do it, without taking an overt or covert vote. Therefore they don't get discouraged if others don't help. Some people simply have an abundance of energy and dedication and some do not. Just expend as much as you have, without looking over your shoulder to see how much others are doing.\\
Not every anarchist can be a bundle of energy and a model of total commitment. Since they consider themselves more important than any group they are part of, real anarchists may withdraw from projects that don't interest them. The group has no claim on anyone; it exists to satisfy the needs of individual members.\\
Groups do not do things. Everything begins with an individual impulse. Even something that looks like teamwork breaks down, on close examination, to the resultant of a number of individual contributions. It all starts, not with a group soul, but with a single person. Each anarchist has the right and the responsibility to define the total picture of the movement, what its principles are, what its aims are, where it is going, what its strategy and tactics should be. No one should sit back and let a few ideologists do all the thinking about the big picture.\\
We are all victims of an authoritarian mind-set that dates back, at least, to the neolithic era. The anarchist movement, little more that a century old, represents a beginning effort by some members of our species to erase that programming and try to think about human problems in a new way. This new thinking and doing, whatever it may become, will not originate with leaders or groups. It will come from individuals, from the voice and the light within.