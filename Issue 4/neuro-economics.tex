\chapter{A Few Blunt Statements about Neuro-Economics}
\chapterauthor{Robert Anton Wilson}

Society derives from sex, from reproductive relationships. Mammalian pair-bonds and pack-bands (imprinted emotions of affection-and-trust) held the first human bands and tribes together as working units. At the canter, the hub, stood the orgasmic tenderness --- the shared love of the genital embrace in the mating act --- and out of this radiated the ``sublimated" tenderness of parent-infant, brother-sister, uncles and aunts and grandparents, the whole ``extended family" or hunting/food-gathering band.\\
The conquering State, and the subsequent fission of society into separate classes of privileged and deprived, created poverty. Poverty as a human institution derives from conquest, from the establishment of government (the invading warrior band, remaining to rule that which they have conquered) and from the institution of ``laws" to perpetuate the class division between Invaders and Invaded.\\
The human, like any other primate, contains neuro-genetic circuits ready to be imprinted by pair-bonds and pack-bonds. The evolutionary purpose of these bonds remains classically mammalian: They insure bio-survival and pack-status. They also program the Seed into heterosexual-reproductive behaviors necessary for the survival of the pack, which in turn provides bio-survival security for future generations.\\
The rise of the conquering State, the feudal State and eventually the modern Capitalist State has progressively undermined and subverted the tribal pack-bond (the ``extended family"). In the most advanced Capitalist nation, U.S.A., little pack-bonding remains. Hardly any U.S. citizens will stop for hitch-hikers, give to beggars on the street or even trust their neighbors. Many don't even \emph{know} their neighbors. Normal pack-bonding behaviors of trust, charity, affection, etc., still found in the Feudal nations, have atrophied here. The celebrated ``anomie," ``anxiety," ``alienation," etc. of Capitalist society begins from this lack of normal pack-bonding.\\
The circuits which normally imprint a pack-bond still survive, ethologically speaking. (In psychological language, the same thought would be expressed by saying that the need for bio-survival security still survives.) This mammalian constant must be satisfied, and in an abstract society the satisfaction becomes abstract.\\
\emph{Paper money becomes the bio-survival imprint in Capitalist society}.\\
William S. Burroughs has compared Capitalism to heroin addiction, pointing out the terrible parallels: The junkie must have regular doses; the Capitalist citizen must have a regular money-fix. If junk is not available, the addict becomes a spasmodic bundle of anxieties; if money is not available, the Capitalist citizen goes through similar withdrawal-trauma. When junk becomes scarce, junkies behave desperately, and will steal or even kill. When money becomes scarce, Capitalist citizens will also rob or kill.\\
Opiate drugs, according to Dr. Timothy Leary, function as bio-survival circuit neurotransmitters. That is, they activate neural networks keyed to the mother-infant bond. (In the terms of pre-neurological Freudian psychology, the junkie in the arms of Mother Opium regresses to infantile bliss.) In the context of a society without the normal mammalian pack-bond, similar imprinting occurs on money, by conditioning learned associations onto the infantile reflexes. The Capitalist citizen learns neurologically that \emph{money}=\emph{security} and \emph{lack of money}=\emph{insecurity}.\\
Infantile separation-anxiety (the fear of losing the all-providing Mother) became generalized to tribal separation-anxiety quite early in hominid evolution. The person thrown out of the tribe for deviant or anti-social behavior experienced \emph{real} survival anxiety. (The tribe, in primitive conditions, has much greater survivability than the lone individual. Ostracism, then, usually meant death, just as ostracism-from-Mother can mean death to the infant.)\\
Since money has replaced the tribe in Capitalist society, the majority of citizens have imprinted onto money the traditional mammalian emotions of the infant-mother and individual-pack survival bond. This depends on conditioned associations through real deprivation experience. Before the rise of Welfarism, people did die of money-withdrawal in Capitalist society, in large numbers; and it still happens, occasionally, among the very ignorant, the very timid, the very old. (E.g., and elderly couple froze to death in Buffalo, a few years ago, when they were unable to pay their utility bill and the local monopoly shut off their heat, in January.)\\
The frequent European observation that Americans ``are money-mad" merely signifies that Capitalist abstraction, and decline of the tribe, has advanced further here than in European Capitalist states.\\
The American, deprived of money, lurches about like a frenzied lunatic. ``Anxiety," ``anomie," ``alienation," etc. increase exponentially, reinforced by real security deprivations. The poor in less abstracted societies share a pack-bond and ``love" each other (on a village level.) The poor in America, lacking the pack-bond, hooked only on money itself, hate each other. This explains the paradoxical observation of many commentators that poverty retains dignity and even some pride in traditional societies, but appears dishonorable and shameful here. Indeed, the American poor not only hate each other; often, perhaps usually, they hate themselves.\\
These facts of neuro-economics have been so charged with pain and embarrassment that most Americans will not discuss them at all. The sexual prudery of the 19th Century has become money prudery. People will talk in the avant third of the population anyway, quite explicitly about the fetishistic aspects of their sex imprints (``I get off on wearing my wife's underwear during the foreplay," or whatnot) but equal frankness about our money-needs freezes the conversation and nay empty the room.\\
Behind superficial pain and embarrassment lies mammalian terror: bio-survival anxiety.\\
The mobility of modern society escalates this money-anxiety syndrome. During the 1930s depression, for instance,  many grocers and other ``corner stores" allowed customers to run up quite large bills, over periods of months sometimes. This was based on the last tattered fragments of the traditional tribal bond and the fact that everybody still knew everybody in the neighborhoods of those days (40 years ago.) Today, it would not happen. We live, as one novel said, in a ``world full of strangers," where, as the Living Theatre says, ``You Can't Live Without Money."\\
In the opening chapter of \emph{The Confidence Man}, Melville contrasts the ``religious nut" who carries a sign saying ``LOVE ONE ANOTHER" with the merchants whose signs say, ``NO CREDIT." The irony was meant to reflect on the uneasy mixture of Christianity and Capitalism in 19th-century America --- but Christianity, like Buddhism and the other post-urban religions, appears to be largely an attempt to recreate the tribal bond on a mystical level within ``civilized" (i.e., Imperialist) times. Welfarism represents the State's attempt to counterfeit such a bond (in a stingy and paranoid fashion, in the spirit of Capitalist law.) Totalitarianism appears as the eruption, in fury and desperation, of the same endeavor to convert the State into a tribal nexus of mutual trust and bio-survival support.\\
The dawn of libertarian philosophy in America featured two tendencies which modern libertarians have neglected --- unwisely, if the above analysis prove sound. I refer to the emphasis on \emph{voluntary association} --- retribalization on a higher level, through shared evolutionary goals --- and on \emph{alternative currencies}. The former of these ideas appears prominently in Warren, Greene, Spooner and Tucker, among others; the latter in all the above and in Dana, Ingalls, C.L. Schwartz, Joseph Labadie, Bilgrim, Levy, etc.\\
Voluntary associations or communes without alternative currencies quickly become re-absorbed into the Capitalist cash nexus. Voluntary associations with alternative currencies openly declared get ground up in the Courts and destroyed. Voluntary associations using covert or secret currencies, as in \emph{Illuminatus!} may actually exist, to judge from hints or codes in some right-wing libertarian publications.\\
No form of libertarianism or anarchism (including anarcho-capitalism and anarcho-communism) can successfully compete with Welfarism or Totalitarianism, under present conditions.\\
Current Welfare practices emerged from 70 years of struggle between Liberals and Conservatives; the Conservatives won most of the battles. The system functions so as to heighten the addiction syndrome. The recipient gets a small fix at the beginning of the month, nicely calculated to support one extremely frugal miser until about the 10th of the month. Through hard experience; SHe learns to make this last until the 15th, maybe even to the 20th. The rest of the month is experienced as acute bio-survival anxiety. This deprivation period, as any pusher or Skinnerian conditioner knows, maintains the whole cycle. On the first of the next month, another money-fix is allowed, and the whole drama begins anew.\\
The Welfare rolls steadily increase, since --- even with the most bumbling inefficiency and redundancy --- the tendency of industrialism remains, as Buckminster Fuller says, to do-more-with-less and omni-ephemeralize. Each decade, fewer will have jobs and more will be on Welfare. (Already, one-half of one percent own 70 percent of the wealth, leaving 99.5 percent to compete violently for the remainder.) The end result could become a totally conditioned society, entirely abstract, motivated only by neurochemical money addiction.\\
To measure our advance toward that condition, imagine vividly what you would do and feel if all your money and sources of money disappeared tomorrow.\\
It is important to bear strongly in mind that we are still discussing \emph{standard mammalian behaviors}. In recent research, chimpanzees have been trained to use money. The reports indicate that they developed normal ``American" attitudes toward the mysteriously powerful tokens.\\
The Illuminati pyramid on the dollar bill, like the similar ``magick" emblems of the \emph{Fleur-de-Lys}, Swastika, Two-Headed Eagle, Stars, Suns, Moons, etc. with which other nations have seen fit to festoon their State currencies or documents, is intrinsic to the ``spookiness" of the whole monopolization of \emph{mana} or psychic energy by the State. Here are two pieces of green paper; one is money, the other is not. The difference is that the former was ``blessed" by the wizards in the Treasury Building.\\
The Capitalist worker lives with the sane perpetual anxiety as the opiate addict. The source of bio-survival security, the neurochemistry of feeling safe, is hooked to an external Power. The conditioned chain \emph{money=security, no-money=terror} is reinforced continually by seeing others ``fired" and fallen by the wayside. Psychologically, this state may be categorized as \emph{chronic low-grade paranoia}. Politically, the manifestation of this neurochemical imbalance is known as Fascism. The Archie Bunker/Adolph Schickelgruber/Richard Nixon mentality.\\
As Leary says, ``Fear and rage restrictions on freedom now dominate our social life... fear and restricting violence can become addictive kicks, reinforced by schizophrenic policy-makers and an economic system which depends upon restricting freedom, and upon the production of fear and the inciting of violent behavior."\\
In Desmond Morris's perfect metaphor, the naked ape behaves exactly like a zoo animal: despair is the essence of the cage experience. In our case, the bars are intangible; imprinted gave-rules --- Blake's ``mind-forg'd manacles." \emph{We are literally being robbed blind. We have literally taken leave of our senses.} The conditioned token, the symbol money, controls our mental well being.\\
This appears to be what Norman O. Brown is groping to say in his Occult-Freudian tomes about our ``polymorphous perversity" (natural body-rapture) being destroyed in the process of conditioning sublimated sex (group-bonding) onto social games like money. The Resurrection of the Body that Brown foresees can only happen through neurosomatic mutation, or as Leary calls it, Hedonic engineering. Historically, the only groups that have ever managed to detach themselves effectively from the social game-anxiety have been (1) absolutely secure aristocracies, free to explore the various ``mental" and ``physical" pleasures, and (2) communes of shared voluntary poverty --- a form of retribalization by sheer determination.\\
Libertarians, like other idealists and malcontents of Left and Right, generally suffer a wounding sense of the ghastly chasm between their evolutionary goals and the present grim reality. This vastly complicates the resolution of their own money-anxiety syndrome, with the result that virtually all of them feel intense guilt about the ways they acquire the money necessary to survive in the domesticated-ape world around us.\\
``He has sold out," ``She has sold out," ``I have sold out," are accusations heard daily in every idealistic clique.\\
Any way of ``making money" automatically opens one to guilt-inducing vibrations from one faction, while it paradoxically spares one from further guilt-inducing vibrations from another faction. Catch-22, the Double Bind, the Snafu Principle, etc. are merely extensions of the basic neuro-economic trap: You Can't Live Without Money.\\
As Joseph Labadie concluded, ``Poverty doth make cowards of us all."\\
There is, ultimately, a pleasure in \emph{enduring} poverty. It is like the pleasure of surviving through grief and mourning and loss; the Hemingway pleasure of standing firm and continuing to fire at a charging lion; the saint's pleasure in forgiving those who persecute him. It is not masochism but pride: I have been stronger than I thought I could be. ``I have not wept nor cried aloud." This is the joy Neitzsche and Gurdjieff found, in ignoring their cruelly painful illnesses and writing only of the ``awakened" state beyond emotions and attachments.\\
Right-wing paranoia about paper money (the various conspiracy theories about how the supply and withdrawal of money is manipulated) will always remain epidemic in Capitalist society. Junkies have similar myths about the pushers.\\
It is real food, real clothing, real shelter that are threatened when money is removed, even briefly, and it is real deprivation that occurs when money is removed for any length of time. The domesticated ape is trapped by a game of mental symbols, but the trap is deadly.\\
There is some kind of masochistic pleasure in continuing the analysis of a painful subject into every byway and intricacy of its labyrinthine torments. There is something of this beneath the ``objectivity" of Marx, Veblen, Freud, Brooks Adams. ``As bad as it is, we can at least look at it without screaming," such writers seem to be assuring us, and themselves.\\
``Only those who have drunk from the same cup know us," said Solzhenitsyn. He was talking about prison, not poverty, but the two are alike in being traditional punishments for dissent. One takes pride in having borne them, if one survives at all.\\
A popular opinion suggests that the counter-culture of the 1960s was beaten to death by cop's clubs, drug busts other direct violence. My impression is that it was simply 
starved out. The money was cut off, and after sufficient deprivation the survivors crawled aboard the first Capitalist life-raft that was passing.\\
Capitalism, Jack London wrote, has its own heaven (wealth) and its own hell (poverty). ``And the Hell is real enough," he added, from bitter experience.\\
Fatherhood is problematical at best, but becomes a hero's task under Capitalism. When the money supply is cut off, the father of a family in U.S. today experiences multiplied anxiety --- fear for self, fear for those who love and trust one. Only the captain of a sinking ship knows this vertigo, this wound.\\
To survive terror is the essence of true Initiation. For they live happiest who have forgiven most, and, as Neitzche said, anything that doesn't kill me makes me stronger.
\begin{flushright}
April 26, 1976
\end{flushright}