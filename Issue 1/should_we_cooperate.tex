\chapter{Should We Cooperate?}
\chapterauthor{Jim Bumpas}

The above question is raised in the context of working with organizations formed upon non-, or maybe anti-anarchist perspectives in joint action upon projects of general popular interest and concern.\\
The gut-level reaction of many anarchists, especially in regard to liberal or Communist groups, is to trash them whenever they ask for our cooperation or when they try to recruit new members. We don't notice them most of the time, but when we do notice them, we first want to deal them the same kind of blow we want to deliver to the corporate establishment. This is all the more tempting because of their relative vulnerability compared to ITT of General Motors.\\
Under certain conditions, my advice is to cooperate with both Marxist-Leninist and liberal groups. My conditions may seldom be met, however. First of all, any cooperation must be only for the limited purpose of furthering a current popular project, such as agitation and organizing public protest (i.e., anti-war or anti-high prices demonstrations).\\
Secondly, the other organizations must recognize and tolerate the organizational integrity of the anarchists and not insist that participating anarchists render their perspectives invisible. Where Marxists or liberals are strong they usually try to insist that all who wish to work on a project which is ``their property" must join their organization and accept its dominant leadership.\\
This proprietary attitude towards popular issues leads them to some of the following tactics in order to compel subservience to their leadership: ``What are you, a wrecker? Do you want to cause disunity and lessen our chances of possible success?" Or, if there is already a coalition of Marxist-Leninist and liberal groups, they will still accuse you of ``wrecking," since anarchists will surely alienate some other members of the coalition. Don't give in to this tactic. The argument cuts both ways. Maintain your independence.\\
Of course, without anarchist organization, you are more or less at the mercy of this tactic. Your energy and commitment to the issue will lead you to contribute much of the effort which will strengthen the organization of the Marxist-Leninist group and increase the future dominance over the local movement.\\
A third condition must be tolerance by the other groups of your expression of anarchist perspectives and of your independent criticism of their perspectives and practice. This will again be described as wrecking, or some similar epithet, but it is necessary to maintain our integrity and to demonstrate that our cooperation is not slavish adherence to dogma which we abhor, but only an ad hoc cooperation for the purpose of the immediate project. Any other course of action would be dishonest and misleading to the very people who have generally misunderstood anarchism. If it is ``wrecking" it is only because Marxist-Leninist leaders fear the effect our perspectives might have on their otherwise docile sheep.\\
If these conditions are refused, it does not preclude independent action on the project, especially if it is important enough to you. If your anarchist group is well enough organized, you will attract independent support of your own, forcing the recognition of hostile groups who are working on similar projects. A group I was a member of in college was forced to take this course because local Marxist-Leninist groups were convinced they had a proprietary interest in the anti-war movement and in any strike-support activity. When our group began conducting our own activities against the war and  in support of various strikes, our strength in both numbers and energy expanded rapidly. Then the Marxist-Leninist and liberal groups condescendingly began to invite us to participate equally in ``their" demonstrations. Anarchist perspectives soon began to excite people who previously had equated anarchy with chaos, anarchism with terrorism, anarchists with bomb-throwers. Only the transient nature of student populations reduced the effectiveness of this group, which still exists in a semi-dormant state after six years.

\section*{Why cooperate?}

All right, now that I've answered the first question affirmatively, and have discussed some of the mechanics of cooperation, why go to all the trouble? Why not trash all enemies or probable enemies whenever and wherever we are confronted with them? The question contains a suggestion of its own answer. Our enemies, both potential and actual, are many and we are few. I alluded to another part of the answer at the beginning of this discussion. The corporate establishment which dominates our culture and society, and which controls the government and its means of repression, including the military and police forces, media, education, etc., are vastly more powerful at this point that all of the Marxist-Leninist and liberal groups who might wish to eradicate anarchists at some time in the future. For purely strategic purposes, cooperation with the enemies of our most powerful enemies is called for.\\
While such cooperation will inevitably strengthen the Marxist-Leninists and liberals relative to the corporate establishment, the conditions I've described should ensure that our own strength should grow apace, if not more rapidly than theirs. At least we will not be swallowed up whole as the Bolsheviks swallowed anarchists in Russia. I realize my conditions for cooperation (not collaboration) are similar to those conditions which operated in revolutionary Spain in 1936--39, but I suggest that these conditions were not a factor in the anarchist defeat there.\\
Cooperation with other opponents of the corporate establishment will help us to create the conditions for a greater flowering of our energies and perspectives. If we continue to allow the biggest oppressor to pit all of its opposition against one another, the status quo stays in balance. It is to our advantage to tip the status quo out of balance. That is my hypothesis. Any comments?