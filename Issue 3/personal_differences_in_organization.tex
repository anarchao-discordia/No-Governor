\chapter{The Role of Personal Differences in Organization}
\chapterauthor{Jim Bumpas}

Question: Why do anarchist organizations suffer disruption and many times suffer dissolution because of destructive personality clashes? This question is not dismissed by recalling that anarchist organizations aren't the only ones which suffer personality clashes.\\
I think I have some insight into at least two related causes. One cause is the social organization of people and things in our society which has concentrated great amounts of power, wealth and prestige into various centralized focuses. There is a whole spectrum from large, manipulative concentrations, down through mostly small, imitative concentrations which mostly provide practice for the ``real" competition in the large powerful organizations. These imitative organizations include especially student governments at schools and lodge and ``service" club organizations for working people.\\
These organizations, large and small, all offer the prizes of wealth, power and prestige in varying amounts, in addition to the enjoyment of comradeship and group effort which may be found in many organizations. In fact, these material rewards are distinct from, and not at all related to the purposes of the collective work of most organizations like political parties and non-profit organizations.\\
Another cause is the destructive social conditioning we all suffer which makes us want to scratch, bite and claw our way to the top of any of these organizations. This conditioning adequately serves the control and domination purposes of establishment organizations. The resources they control and the coercive machinery of the state are available to them for protection. So the competitors for position are forced to conform to the dog-eat-dog way of life. Slight personal differences basic to our individuality are struck upon and exaggerated both to justify one's fight against someone in his way and to serve as a tool to beat the competitor down.\\
Anarchist organizations control no such set of socially-sanctioned prizes. The reward of collective work in such organizations is just that collective work and the comraderie of closeness to persons of similar social perspectives.\\
However, all of us enter anarchist organizations carrying various amounts of baggage containing this ``scratch, bite, and claw" conditioning. We are all affected somewhat by the desire for material rewards. These two factors produce disruption and destruction to anarchist organizations.\\
We practice in our organizations many of the same destructive relationships found in the society we desire to alter. We strike upon some real or fancied minor difference in personality or approach and try to elevate it into a grand ideological distinction. This leads one to try to purge the other, or bend the other to conform to one's criticism. Or, failing all that, one can split the group and lead your followers out.\\
Even in a group whose members share similar or perhaps the same perspectives --- even when method and practice are exact --- a minor shift in emphasis between two or more persons can give rise to the destructive abandoning of the purposes of the group's collective work. One person begins to play down media/theater type actions and stresses the importance of concentrating all activity on contacting people where they work, live or play in order to provide organizational tools for the expression of and achieving their desires. Another downplays such activity and stresses the consciousness-altering value of shocking media/theater actions. Another plays down both and stresses the need for direct, violent action against persons and institutions identified as oppressive.\\
These differences arise mostly because of differences in individual personality and in personal evaluations of current circumstances, both of society in general and the organization and its individual members. But instead of mutual recognition of each other's mutual dignity and personality, and instead of attempting to analyze clearly personal interrelationships and current circumstances, many times these differences are elevated into grand ideological distinctions. Instead of discussing social conditions and organizational resources in order to adopt the tactics which most fit present capacities for action, our mutual perspectives are abandoned --- and real people are abandoned, too --- in order to develop, preserve and protect the rarified ideological purity of one person or another. The exaggerated ideological/personal differences become so important that group effort suffers and withers away. The group splits or shatters and constructive work ceases until everyone begins again, almost from scratch in some cases. Sometimes, constructive group work only begins to grow again as a result of another split in another group. Old Splittees  join new splittees in common aversion to the splittees of the other ``side."\\
In anarchist organizations there is no position equivalent to chairman of the central committee or president. Nor are there big salaries given as plums to the victors in the struggle between personalities. So the groups divide, or they fail to collaborate or affiliate in solidarity for the common project. Each little groupuscule isolates itself from the other and creates for itself much of the alienation felt by anarchists toward some other anarchists.\\
I'm convinced it's a mistake and a violation of our perspectives to try to correct the problem by creating some ``anarchist" equivalent to the hierarchy or salaries of the authoritarians. But we cannot ignore the problem. Perhaps clear analysis of some of the causes for disruption in our work will allow us to minimize the destruction caused by these personality clashes. Comparison of our forms organization to those hierarchical forms found in society around us will aid us in this analysis. Why do we differ from them? Why must authoritarians maintain hierarchy and how does it further their goals? Can we organize society without hierarchy and the motivations which impel people to work within authoritarian forms?\\
It appears to me that we must show some spectacular success in organizing ourselves along anti-authoritarian lines before we can make a significant advance against the popular authoritarian assumption that nothing gets done unless someone orders others to do it.
