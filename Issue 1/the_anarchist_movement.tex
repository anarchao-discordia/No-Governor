\chapter{The Anarchist Movement --- Dead or Alive?}
\chapterauthor{Arlene Meyers}

Once we get beyond the level of personal hostility in the \emph{SRAF Bulletin}, I think it's time we begin a serious critique of the anarchist movement. The first question is: Are we really a revolutionary movement or not?\\
When non-anarchists express interest in the ideas of anarchist thinkers, the question they ask is, ``How does anarchism work?" When all we have to offer are books on historic struggles or our own splintered and embattled ranks as examples of anarchism at work, we don't make any positive statement on anarchism.\\
From my association with Chicago anarchists, I would say our primary problem is integrating our politics with our daily lives. Not a matter of assuming an identity as an ``anarchist" --- role-playing at best, confusing revolution with religion at worst --- but as a conscious and deliberate process of creating social change in our own heads. Fucked-up people create fucked-up revolutions. The shape and direction of our revolution is determined by the kind of people we are, how we are growing and changing as individuals, and how we relate to one another.\\
Because so many anarchists I know are basically alienated, anti-social types incapable of genuine interaction with others, the movement reflects this isolation. Many anarchists are well-read, even brilliant in their grasp of history and theory, but because they lack event minimal social skills, they are unable to translate their ideas into collective action. Collective activity (where possible) is centered within small, closed friendship groups. Because these groups are so closed off and ingrown, their level of energy and activity is very limited. Even when the friendship group seeks to involve others in its activity, it is unable to open up its internal social structure to include new members because to do so would require an openness and energy the group does not possess.\\
Thus, we have a closed cycle of alienation (from society at large), exclusive association (with like-minded individuals), and isolation (based upon these ingrown structures). At this point anarchism is not a social movement but an anti-social movement.\\
Chicago anarchists --- other than a few exclusive friendship groups --- do not even care to associate with one another, because our attempts at open meetings several years ago became lifeless weekly rituals which anyone could come into and freely disrupt everyone else (and frequently did). Individuals with something interesting to say (and well worth listening to), and people who wanted to learn about anarchism and people who simply desired a social group were all lumped together in one frustrated mass. The logical thing might have been to separate into specific groups, but since we really had no clear idea of our own needs, I think we feared separation might lead to further isolation. In the end the group disintegrated anyway --- lack of direction more than anything else, I think.\\
Even now, those of us involved in activity in Chicago do not have any sort of \emph{open}, regular forum where we can exchange ideas, offer mutual aid and support or even keep each other informed, and this lack of communication seriously threatens our present level of activity and our continued growth; Because of previous hassles with organizing ourselves, I understand the reluctance of many anarchists here to have open meetings again, but ignoring problems does not make the problems go away.\\
Good communication is the lifeline of non-hierarchical organization. In Chicago, lines of communication are the ``property" of specific publications, groups and privileged individuals (e.g., \emph{Industrial Worker}, \emph{Black Cross Bulletin}, \emph{Siren}, I.W.W., Solidarity Bookshop). It was only when I began publishing \emph{Siren} that I was allowed access to privileged information withing the movement itself through exchange papers, personal correspondence and other resources. Unfortunately, there was no way for me to pass this privileged information along to other people because my efforts in organizing an anarcho-feminist group here were met with both hostility and indifference, and infrequent Solidarity Bookshop meetings (when called) were confined to bookshop business itself, discussions on stock, bills, etc.\\
Aside from necessary security precautions, news and information should always be offered freely, and not withheld as a condition for inclusion in the group\footnote{I mean, trading labor and loyalty for access to vital information (as a lot of ``revolutionary" groups do) is not very revolutionary. Solidarity people didn't even offer this much --- they are \emph{so} exclusive, I have no idea what is their criterion for inclusion in the inner circle. Which means that if there is no interaction within the collective structure (and collective identity) there is usually a big turnover in personnel.} (e.g., payment of dues, membership, building organizational strength, etc.). Lately I've stayed away from groups involved in social action because they demand some sort of commitment in the form of membership, dues, voluntary labor, etc. before offering any information, and I refuse to involve myself with people who play these games. Don't expect people to become involved if you refuse to open up to \emph{them}. People do not become involved in revolutionary activity in order to be as isolated as they are elsewhere: we become involved because we seek new ideas and associations, not merely to exchange one form of alienated activity for another.\\
Too often, anarchist activity is confined to the individual's own particular level of energy and resources and this usually means a lot of fragmented, duplicated efforts. Collective activity requires a level of social consciousness and cooperation the self-centered egotist does not possess. Collective activity should not require a submersion of the ego, but it does require a free exchange of ideas and energy. Insecure people who are jealously concerned only with their own ego needs can never participate freely in collective activity, because they become competitive in their drive for ``property" --- in this case, power and recognition --- in order to assure themselves a secure place in the collective structure. It's one thing to work with dynamic and energetic individuals; it's another thing to work with insecure people who engage in ruthless competition; backbiting and power-plays in their endless search for status and prestige.\\
Aside from a few enlightened individuals, too many anarchists I know are closed to struggle because they prefer to project a cool image based upon knowledge and experience they do not possess. This is revolutionary role-playing, and it's a mind-fuck. When people claim to be anarchists, and only offer theoretical abstractions or vague generalizations in response to specific problems, I know I'm dealing with people who not only have no answers but who refuse to commit themselves to an honest search for answers because it might reveal their inadequacy as ``revolutionaries." It is this unwillingness to struggle, and this moral cowardice in dealing with real problems which seriously flaws the anarchist movement of today.\\
Too many of us see ``struggle" solely in militaristic terms: bombs, bullets, sabotage, terrorism, etc. Yet struggle also means dealing with one's own fears and insecurities, such as racist or sexist attitudes. For many of us, it is easier to confront a line of riot cops than to deal with the complexities of personal relationships, yet power games and authoritarian attitudes begin in our own little heads, and our heads are really the first battleground where we initiate the struggle against the state. We must free ourselves of our fears and hypocrisies and power relationships because only a free people can create a revolutionary movement capable of transforming patriarchal, authoritarian structures into the free society we desire.\\
A revolutionary movement is only a vehicle to freedom, and not freedom itself. It must be both open and viable; a place to discuss new ideas and to experiment with new forms and structures; a vehicle for dialogue and discussion. It's nice to have faith in the spontaneity of the masses, but I find that waiting for the masses to respond only means a repetition of past failures. During 1969 and 1970 we witnessed the sudden growth of both the anti)war and women's movements, and the frustration and confusion we experienced the were due to greatly increased expectations on the part of too many people who literally expected instant solutions to long-standing problems. Luckily the women's movement has survived this onslaught of mass media publicity, since it has organized itself around women's needs (it is deeply rooted in women's lives), but the general movement has foundered for lack of direction and constructive activity.\\
At this time the anarchist movement is ineffective because it has yet to resolve basic problems of organization, communication and information flow, or learned how to settle internal disputes. Too much of our literature lacks fresh, creative or original thinking, and we are too little concerned with our internal problems, and too much of our association is marred by personality conflicts and petty disputes.\\
Much of classical anarchist thinking is visionary, inspirational, deeply ethical. Yet we need to develop a better understanding of how to actually create that revolution we desire; a critique of our daily lives. Radical feminism has come closest of any revolutionary movement to understanding the socialization process which stifles our ability to think and act freely, yet too many anarchist men (and women) continue to see feminism only withing the context of anti-sex (neo-puritans), role reversal or neurotic bitching, and are so threatened by the challenge to accepted behavior patterns and thinking they prefer to ignore or trivialize its meaning.\\
I am not convinced that anarchism works when I see a movement so loaded with hypocrites, moral cowards and intellectual snobs incapable of integrating their principles with their lives, of taking a principled position and of translating theory into action. I'm not asking us to get out the guns and ammunition (a loser's game, anyway), I'm asking us to quit hiding behind a wall of rhetoric and naivete and to work out some practical ways of resolving our problems, initiating meaningful activity and seriously creating a revolutionary movement.\\
Otherwise we can all stay home and watch the empire crumble on TV. Unless, of course, the energy crisis gets too close to home.