\chapter{The Dialectic of Morality}
\chapterauthor{Josh the Dill}

Why are we forever patting ourselves on the back because we practice virtue and pursue ideals? Why are we condemning each other all the time for being immoral? Because, we say, our efforts at morality, virtue and idealism have raised us above the level of the beasts. Some justification! Who, when he or she really considers the matter, would really want to be above the level of the beasts, and why? Beast do not destroy their own environment, do not massacre millions of their own species, do not practice sacrifice of their own kind to propitiate spooks. The human race likes the idea of feeling superior to animals because it is on an ego trip. We vaunt our civilization, the extent to which we have created defenses against the sort of suffering animals must endure. Bur our ideals and values have produced their opposites. If there were no goodness, there would be no evil. No law and order, no crime. Every figure calls up a ground, every black implies a white. The distinctions between black and white and the rest are necessary, but so is recognition of fundamental one-ness. This point applies to violence and nonviolence. Making nonviolence a moral absolute will provoke violence. Violence, in turn, generates nonviolence as a response. Martin Luther King's nonviolent march from Selma to Montgolery was protected by Federal paratroops.

\newpage
\blockquote{``Utopia must spring in the private bosom before it can flower in civic virtue, inner reforms leading naturally to outer ones. A man who has reformed himself will reform thousands.
\par\begin{flushright} --- Paramahansa Yogananda, \emph{Autobiography of a Yogi}\end{flushright}
}